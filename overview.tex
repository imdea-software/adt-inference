\section{Overview of an ADT Inference Algorithm}
\label{sec:overview}

Our basic approach to inferring ADT specifications, as outlined by the abstract
algorithm in Algorithm~\ref{alg:abstract}, is to identify a finite set of
sequential execution histories which capture all of the reasons for which a
sequence could be considered invalid, according to the ADT of a given reference
implementation. These sequences thus serve as patterns indicating violations in
the sequences which contain them. Thus the linearizations which exclude all
violation patterns are considered valid.

\begin{algorithm}[t]
  \SetAlgoLined
  \SetKwInOut{Input}{input}
  \Input{A reference implementation $\mathcal{I}$}
  \KwResult{A formula representing the ADT of $\mathcal{I}$}
  patterns $\gets$ $\emptyset$ \;
  \For{each sequential history $h$}{
    \uIf{$*$}{
      break
    }
    \uElseIf{$h$ is \emph{executable} with $\mathcal{I}$}{
      continue
    }
    \uElseIf{$h$ is \emph{redundant} with patterns}{
      continue
    }
    \Else{
      add $h$ to patterns
    }
  }
  \Return \emph{exclusion} of patterns
  \caption{Abstract algorithm for symbolic ADT inference.}
  \label{alg:abstract}
\end{algorithm}

As an example of this algorithm at work, consider the sequential histories
listed in Figure~\ref{fig:patterns}. These sequences arise from an enumeration
of all $202$ possible method invocation sequences\footnote{Here we consider
equivalence among sequences which are isomorphic up to renaming of data values,
e.g.,~to avoid considering {\tt add(1); add(2); remove => 1} and {\tt add(3);
add(2); remove => 3} as distinct. The notion of \emph{matching} introduced in
Section~\ref{sec:histories} provides a clean formal treatment.} of length at
most $4$ of an object with add and remove methods for which remove can return
empty. The $31$ sequences which are executable by a correct reference
implementation of an atomic queue — i.e.,~with consistent return
values for each invocation — are discarded. The remaining $171$ sequences are
invalid, $164$ of which are redundant with the seven {\tt PATTERNS} listed on
the left-hand side of Figure~\ref{fig:patterns}. Seven of these redundant
sequences are shown on the right-hand side of Figure~\ref{fig:patterns}. For
example, the first five sequences on the right, as well as the last, are
redundant with the first on the left since they each describe a violation in
which a removed element was never added. The sixth sequence on the right is
redundant with the second on the left since they both describe a violation in
which remove returns empty before previously-added elements are removed. Other
represented violations include removing an element before it is added, and
removing elements in the opposite order in which they were added. Finally, the
exclusion of the computed pattern set can be expressed as a conjunction of
formulas in first order logic describing the exclusion of each individual
pattern. For example, the first pattern could be described by the formula
\begin{align*}
  \exists o.\
    {\sf method}(o) = {\tt remove} \land {\sf unmatched}(o)
\end{align*}
describing an unmatched remove operation. The formula which excludes all
violation patterns is satisfied by an invocation sequence if and only if that
sequence is admitted by the given reference implementation’s implicit ADT.

\begin{figure}[t]
  \begin{minipage}[b]{0.49\linewidth}
    \begin{verbatim}
[1:X] remove => 1  #
---
[1:1] add(1)           #
[2:2] remove => empty    #
---
[1:2] remove => 1  #
[2:2] add(1)         #
---
[1:1] add(1)       #
[2:2] add(2)         #
[3:2] remove => 2      #
---
[1:1] add(1)       #
[2:1] remove => 1    #
[3:1] remove => 1      #
---
[1:1] add(1)           #
[2:2] remove => empty    #
[3:1] remove => 1          #
---
[1:1] add(1)       #
[2:2] add(2)         #
[3:2] remove => 2      #
[4:1] remove => 1        #
    \end{verbatim}
  \end{minipage}
  \hfill
  \begin{minipage}[b]{0.49\linewidth}
    \begin{verbatim}
[1:1] add(1)       #
[2:X] remove => 2    #
---
[1:X] remove => 1  #
[2:X] remove => 1    #
---
[1:X] remove => 1      #
[2:2] remove => empty    #
---
[1:1] remove => empty  #
[2:X] remove => 1        #
---
[1:1] add(1)       #
[2:2] add(2)         #
[3:X] remove => 3      #
---
[1:1] add(1)           #
[2:2] add(2)             #
[3:3] remove => empty      #
---
[1:1] add(1)       #
[2:1] remove => 1    #
[3:X] remove => 2      #
    \end{verbatim}
  \end{minipage}
  \caption{Invalid sequences according to the atomic queue ADT. Histories
  on the right are each redundant with some on the left; those on
  the left constitute a complete set. The “\#” symbols depict the time intervals
  spanned by operations horizontally, and the “{\tt [$i$:$j$]}” notation refers
  to the current line’s operation identifier $i$, and the identifier $j$ of
  its matching operation. Adds and empty removes match themselves, and the
  “{\tt X}” symbol denotes an absent match.}
  \label{fig:patterns}
\end{figure}

The key technical obstacle in realizing Algorithm~\ref{alg:abstract},
which we overcome in Section~\ref{sec:patterns}, is
classifying violations into a finite set of patterns, rendering the
remaining invalid sequences redundant. The symbolic representation which
excludes a given set of patterns is relatively straightforward to construct, and
is given in Section~\ref{sec:formula}. Note however that the termination of this
abstract algorithm is nondeterministic, and thus the inferred ADT specification
is generally not the strongest possible in the sense that it may not exclude
certain invalid linearizations. Nevertheless, specifications inferred by this
algorithm are sound, in the sense that when modular program reasoning with
inferred specifications succeeds, correctness follows. In any case, in
Section~\ref{sec:algorithm} we give conditions under which our refinement
of this abstract algorithm is both sound and complete, and show that these
conditions hold for naturally-occurring ADTs, thus resulting in sound and
complete symbolic ADT specifications. The remainder of this article develops
the technical machinery required to realize this abstract inference algorithm.
