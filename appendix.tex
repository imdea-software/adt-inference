%!TEX root = draft.tex
\appendix
\section{Proof of Lemma~\ref{lem:wqo}}

It is well-known that every potentially infinite set of elements has a finite set of minimals w.r.t. a well-quasi-ordering.

We first prove that the embedding order, defined hereafter, is a wqo on width-bounded
labeled interval orders.
Let $\Sigma$ be a finite alphabet. A labeled interval order is a triple $(A,\leq,\ell)$, where $(A,\leq)$ is an interval
order, and $\ell:A\rightarrow \Sigma$ is a labeling function. The embedding order $\subseteq$ between labeled interval orders is defined as
usual by:
\begin{itemize}
	\item $(A_1,\leq_1,\ell_1)\subseteq (A_2,\leq_2,\ell_2)$ iff there exists an injective function $h:A_1\rightarrow A_2$ that
	preserves the labeling, i.e., $\ell_1(x)=\ell_2(h(x))$, for every $x\in A_1$, and the order constraints, i.e.,
	for every $x,y\in A_1$, if $x \leq_1 y$, then $h(x)\leq_2 h(y)$.
\end{itemize}

The width of an interval order $(A,\leq,\ell)$ is the maximum number of elements which are mutually unordered.

\begin{lemma}
$\subseteq$ is a wqo on width-bounded interval orders.
\end{lemma}
\begin{proof}
By definition, for every interval order $(A,\leq,\ell)$, there exists a mapping $I$ from elements of $A$ to intervals on $\mathbb{N}$ such that for every $x,y\in A$, if $x\leq y$, then the interval $I(x)$ ends before the interval $I(y)$ (i.e., the upper bound of $I(x)$ is strictly smaller than the lower bound of $I(y)$). %In general, there are more than one mappings $I_A$ that satisfy these constraints. In the following, we assume that the mapping $I_A$ is arbitrary but fixed.
Therefore, in the following, we assume that an interval order is a multiset $\Gamma$ of triples of the form $[i,j,a]$ where $[i,j]$ is an interval on $\mathbb{N}$ and $a$ is a symbol in $\Sigma$.

We actually prove that another order on interval orders, denoted by $\subseteqq$, which is stronger than the embedding order $\subseteq$, is a wqo. The order $\subseteqq$ is defined by: $\Gamma_1\subseteqq \Gamma_2$ iff there exists an injective function $h:\Gamma_1\rightarrow \Gamma_2$ such that
\begin{enumerate}
	\item $h$ preserves the labeling, i.e., for any triple $[i,j,a]$, $h([i,j,a])=[i',j',a]$, for some $i'$ and $j'$,
	\item $h$ preserves the ordering constraints, i.e., for every two triples $[i,j,a]$ and $[k,l,b]$ such that $j< k$, $h([i,j,a])=[i',j',a]$, $h([k,l,b])=[k',l',b]$, and $j'< k'$.
	\item for every two incomparable elements $x$ and $y$ of $\Gamma_1$, if the interval of $x$ starts before the interval of $y$, then the interval of $h(x)$ also starts before the interval of $h(y)$. Formally, for every two triples $[i,j,a]$ and $[k,l,b]$ such that $i\leq k$ (i.e., the first interval starts before the second interval), $h([i,j,a])=[i',j',a]$, $h([k,l,b])=[k',l',b]$, and $i'\leq k'$.
\end{enumerate}
We say that $h$ \emph{witnesses} $\Gamma_1\subseteqq \Gamma_2$.

Assume $\Gamma_1$, $\Gamma_2$,$\ldots$ is a \emph{bad} sequence, i.e., an infinite sequence of interval orders s.t. there exists no $i < j$ with $\Gamma_i\subseteqq \Gamma_j$. Also, assume that $\Gamma_1$ is the minimal size interval order that can start a bad sequence, $\Gamma_2$ is the minimal order that can continue a bad sequence starting with $\Gamma_1$, and so on.

For each $\Gamma_k$, let $Min(\Gamma_k)$ be a triple $[i,j,a]$ such that
(1) $i$ is the minimal lower bound of an interval in $\Gamma_k$, and
(2) $j$ is the minimal upper bound of an interval in $\Gamma_k$ with lower bound $i$.
%In general, this triple is not unique but it’s not important which one we choose.
Also, let $Init(\Gamma_k) = (a_k, P_k)$, where $Min(\Gamma_k)=[i,j,a_k]$, for some $i$ and $j$, and $P_k$ is the multiset of symbols labeling elements of $\Gamma_k$ that are incomparable to $Min(\Gamma_k)$. Note that $P_k$ is bounded since we assume width-bounded interval orders.

The infinite sequence $\Gamma_1$, $\Gamma_2$,$\ldots$ contains an infinite sequence $\Gamma_{k_1}$, $\Gamma_{k_2}$,$\ldots$ which have the same value for $Init$.
For each $\Gamma_k$, let $\Lambda_k$ be the interval order obtained from $\Gamma_k$ by removing the triple $Min(\Gamma_k)$.
By the minimality assumptions, the infinite sequence $\Gamma_1$, $\Gamma_2$,$\ldots$,$\Gamma_{k_1-1}$, $\Lambda_{k_1}$, $\Lambda_{k_2}$,$\ldots$ is not bad.
Therefore, there exists $m<n$ such that $\Lambda_{k_m}\subseteqq \Lambda_{k_n}$.

It remains to prove that:
\begin{itemize}
	\item if $Init(\Gamma_{k_m})=Init(\Gamma_{k_n})$ and $\Lambda_{k_m}\subseteqq \Lambda_{k_n}$, then $\Gamma_{k_m}\subseteqq \Gamma_{k_n}$.
\end{itemize}

Let $h$ be the injective function witnessing $\Lambda_{k_m}\subseteqq \Lambda_{k_n}$. We prove that the extension $h'$ of $h$ between $\Gamma_{k_m}$ and $\Gamma_{k_n}$, defined by $h'(Min(\Gamma_{k_m}))=Min(\Gamma_{k_n})$ and $h'(t)=h(t)$, for all $t\in \Lambda_{k_m}$, witnesses $\Gamma_{k_m}\subseteqq \Gamma_{k_n}$.

Clearly, $h'$ preserves the labeling.
To prove that $h'$ preserves the ordering constraints, let $Min(\Gamma_{k_m})=[i,j,a]$ and $[k,l,b]\in \Gamma_{k_m}$ such that $j< k$. Also, let $Min(\Gamma_{k_n})=[i',j',a]$ and $h'([k,l,b])=h([k,l,b])=[k',l',b]$. Assume by contradiction that $k'\leq j'$. Since $h$ is injective, there exists an element $[e,f,c]\in \Gamma_{k_m}$ incomparable to $Min(\Gamma_{k_m})$ which is mapped to an element $[e',f',c]$ greater than $Min(\Gamma_{k_n})$ (because, by definition, $\Gamma_{k_m}$ and $\Gamma_{k_n}$ have the same number of elements incomparable to $Min(\Gamma_{k_m})$ and respectively, $Min(\Gamma_{k_n})$).
Since $[e,f,c]$ is incomparable to $[i,j,a]$, we obtain that $e\leq j< k$. Also, by the current assumptions, $k'\leq j'<e'$. Therefore, there exist two elements $[e,f,c]$ and $[k,l,b]$ such that the interval $[e,f]$ starts before $[k,l]$ which are mapped by $h$ to the elements $[e',f',c]$ and $[k',l',b]$ such that the interval $[e',f']$ starts after $[k',l']$. This contradicts the fact that $h$ witnesses $\Lambda_{k_m}\subseteqq \Lambda_{k_n}$.

The properties of $Min(\Gamma_{k_m})$ and $Min(\Gamma_{k_n})$ imply that $h'$ satisfies also the third property in the definition of $\subseteqq$.

Finally, since $\subseteqq$ is stronger than $\subseteq$, we get that $\subseteq$ is a wqo on width-bounded labeled interval orders.
\end{proof}

\begin{lemma}

  $\preceq$ is a wqo on width-bounded histories.

\end{lemma}

\begin{proof}

%We prove that $\preceq$ is well-founded, i.e., there is no infinite strictly decreasing sequence $h_1 \succ h_2\succ \ldots$\footnote{We define $\succ$ as usual by $h\succ h'$ iff $h\succeq h'$ and $h\neq h'$.},
%and that $\preceq$ has no infinite antichain, i.e., there is no infinite set $\{h_1,h_2,\ldots\}$ of mutually incomparable elements
%$h_i\not\preceq h_j$ and $h_j\not\preceq h_i$ when $i\neq j$.
%
%Concerning well-foundedness, note that a given history $h_1$ can be weaken only finitely many times by removing matches, duplicates in a match, or happens-before ordering constraints, or by turning completed operations into pending operations. Since we restrict ourselves to width-bounded histories, $h_1$ can also be weaken only finitely many times by adding pending operations (adding a pending operation increases the width of a history by 1).
%
%TODO PROBLEM WITH PENDING OPS

%To prove that $\preceq$ has no infinite antichain,
%we use the following definitions.
Let $h_1 h_2\ldots $ be an infinite sequence of histories. We prove that there exists $i<j$ such that $h_i\preceq h_j$.

The \emph{signature} of an operation $o$ in a history $h$ is the pair $\sigma(o)=(c(o),f(o))$.
Then, the \emph{signature} of a match $\mu=\set{ o' \in O : m(o') = o}$ is the pair
\[
\sigma(\mu)=(\sigma(o),\set{\sigma(o') : m(o')=o, o'\neq o})
\]
containing the signature of the match target and the set of the signatures of the other operations.
The \emph{signature} of a history $h$ is the tuple
\begin{align*}
\sigma(h)&=&(\set{\sigma(\mu): \mu\mbox{ a match}}, \set{\sigma(o): o\mbox{ read-only}}, \\
&& \set{\sigma(o): o\mbox{ unmatched}}, \\
&& \mset{\sigma(o): o\mbox{ pending and non-matched}}).
\end{align*}

The history signature contains sets of signatures
(for matches, read-only and respectively, completed and unmatched operations)
and the multiset of signatures of the pending and non-matched operations.

%Assume by contradiction that $\preceq$ has an infinite antichain.
Since the set of history signatures is bounded (because the set of methods
and the set of pending operations are bounded),
the sequence $h_1 h_2\ldots $ contains an infinite sequence of histories $h_1' h_2' \ldots$ that have
the same signature.

%
%Also,

The \emph{vector} of a history $h$ is the tuple
\begin{align*}
\nu(h)&=&(\mset{\sigma(\mu): \mu\mbox{ a match of $h$}}, \mset{\sigma(o): o\mbox{ read-only}}, \\
&& \mset{\sigma(o): o\mbox{ unmatched}}).
\end{align*}

Let $\leq$ be an order relation on multisets $\alpha:\Sigma\rightarrow\mathbb{N}$ of (match or operation) signatures from a finite set $\Sigma$
defined by $\alpha\leq\alpha'$ iff $\alpha(\sigma)\leq\alpha'(\sigma)$, for every $\sigma\in\Sigma$.
The relation $\leq$ is a wqo and so is the component-wise extension of $\leq$ to history vectors.
Therefore, by known results, the sequence of histories $h_1' h_2' \ldots$ contains
an infinite sequence $h_1'',h_2'',\ldots$ such that $\nu(h_1'')\leq \nu(h_2'')\leq \ldots$.

The \emph{vector} of a match $\mu$ is the pair
\[
\nu(\mu)=(\sigma(o),\mset{\sigma(o') : m(o')=o, o'\neq o})
\]
containing the signature of the match target and the multiset of the signatures of the other operations.

For every history $h$ and match signature $\sigma$, the \emph{$\sigma$-vector} of a history $h$ is the multiset
$
\nu_\sigma(h)=\mset{\nu(\mu): \sigma(\mu)=\sigma}
$
of match vectors of signature $\sigma$.

Let $\leq_v$ be an order relation on $\sigma$-vectors defined by $\nu_\sigma(h)\leq_v \nu_\sigma(h')$ iff
for every match vector $\nu$ in $\nu_\sigma(h)$ there is a match vector $\nu'$ in $\nu_\sigma(h')$
	s.t. $\nu\leq \nu'$ (here, $\leq$ is the order on multisets defined above).
By known results, $\leq_v$ is a wqo.

Let $\sigma_1$, $\sigma_2$,$\ldots$, $\sigma_n$ be the match signatures defined over a set of methods $\mathbb{M}$.
Since $\leq_v$ is a wqo, the sequence of histories $h_1'',h_2'',\ldots$ contains an
infinite sequence $h_{1}^1,h_{2}^1,\ldots$ such that
$\nu_{\sigma_1}(h_{1}^1)\leq_v \nu_{\sigma_1}(h_{2}^1)\leq_v \ldots$.
Then, the sequence $h_{1}^1,h_{2}^1,\ldots$ contains
an infinite sequence $h_{1}^2,h_{2}^2,\ldots$ such that
$\nu_{\sigma_2}(h_{1}^2)\leq_v \nu_{\sigma_2}(h_{2}^2)\leq_v \ldots$.
Applying a similar reasoning for the rest of the match signatures,
we obtain that $h_1'',h_2'',\ldots$ contains an
infinite sequence $h_{1}^n,h_{2}^n,\ldots$
such that
$\nu_{\sigma}(h_{1}^n)\leq_v \nu_{\sigma}(h_{2}^n)\leq_v \ldots$, for every signature $\sigma$.
Recall that we also have that
$\nu(h_1^n)\leq\nu(h_2^n)\leq\ldots$ since $h_{1}^n,h_{2}^n,\ldots$ is a sub-sequence of $h_1'',h_2'',\ldots$.

Viewing histories as labeled interval orders, where the label of an element $o$ is the triple $(c(o),f(o),r(o))$,
we get that the embedding order $\subseteq$ is a wqo on histories. Therefore, the infinite sequence
$h_{1}^n,h_{2}^n,\ldots$  contains two elements $h_{i}^n$ and $h_j^n$ with $i<j$ such that $h_{i}^n\subseteq h_j^n$.
This implies that $h_i^n\preceq h_j^n$, which ends our proof.
\end{proof}
