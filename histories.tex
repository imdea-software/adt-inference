%!TEX root = draft.tex
\section{Histories of ADT Implementations}
\label{sec:histories}

TODO MOTIVATE THIS SECTION

We fix an arbitrary infinite set $\mathbb{O}$ of operation identifiers, and
given sets $\mathbb{M}$ and $\mathbb{V}$ of method names and values. A
\emph{call action} binds an operation $o \in \mathbb{O}$ with a method name $p
\in \mathbb{M}$ and argument value $v \in \mathbb{V}$, while a \emph{return
action} binds an operation $o \in \mathbb{O}$ with a return value $v \in
\mathbb{V}$. An \emph{execution} $e$ is a sequence of call and return actions
where
\begin{itemize}

  \item each operation identifier is used in at most one call action, and in at
  most one return action, and

  \item each return action is preceded by a call action with the same operation
  identifier.

\end{itemize}
An \emph{implementation} $\mathcal{I}$ is a prefix-closed set of executions
which is additionally closed under
\begin{itemize}

  \item appending call actions (of fresh operations),

  \item permuting call actions backward, and

  \item permuting return actions forward.

\end{itemize}
These conditions capture the environment in which implementations execute:
calls are always enabled in their client programs, and thread schedulers may
induce arbitrary delay between implementation code and the associated call and
return actions~\cite{conf/popl/BouajjaniEEH15}.

\begin{example}

  TODO EXAMPLE EXECUTION

\end{example}

In this work, histories are abstractions of executions which retain method
names yet lose the exact argument and return values, and retain the relative
order of operations yet lose the exact sequence of call and return actions.
Formally, a \emph{history} is a tuple $h = \tup{O,<,c,f,m,r}$ where
\begin{itemize}

  \item $O \subseteq \mathbb{O}$ is a set of \emph{operations},

  \item $<$ is a partial \emph{happens-before} order on $O$,

  \item $c: O \to \mathbb{B}$ labels operations as \emph{completed}, or not,

  \item $f: O \to \mathbb{M}$ labels operations with method names,

  \item $m: O \rightharpoonup O$ is a partial \emph{matching} function, and

  \item $r: O \to \mathbb{B}$ labels operations as \emph{read-only}, or not.

\end{itemize}
Non-completed operations are \emph{pending}, and are maximal in the
happens-before order. 

Two operations $o_1, o_2 \in O$ are \emph{identical} when
\begin{itemize}

  \item they have the same labels: $c(o_1) = c(o_2)$, $f(o_1)= f(o_2)$, and
  $r(o_1) = r(o_2)$, and

  \item they have the same matching: either $m(o_1) = m(o_2)$, or both $m(o_1)$
  and $m(o_2)$ are undefined.

\end{itemize} 
The \emph{frequency} of an operation $o$ is the number of operations identical
to $o$, denoted $\freq(o)$. We say that $o$ has \emph{duplicates} when
$\freq(o) > 1$. An operation $o \in \img(m)$ in the image of $m$ is a
\emph{match target}, and the inverse set $m^{-1}(o)$ of a target is a
\emph{match}. We assume $o \in m^{-1}(o)$. A operation $o$ is \emph{unmatched}
when $o$ is completed and $m(o)$ is undefined.

The \emph{width} of a history is the maximum number of its operations which are
mutually unordered, and the \emph{width} of a history set is the maximum width
of its elements. Width-$1$ histories are \emph{sequential}. The \emph{weight}
of a history is the maximum among operation frequencies and the number of
matches. Since the operation identifiers of a history have no intrinsic
meaning, we consider equality between histories up to renaming of operation
identifiers.

\begin{example}

  TODO HISTORY EXAMPLE

\end{example}

Our abstraction of executions relies on correlating the operations associated
with the same values in an execution via the partial matching function of a
history. We construct partial matching functions systematically. A
\emph{matching scheme} $\tup{M,R}$ associates to each execution $e$ with
operations $O$ a partial matching function $M(e): O \rightharpoonup O$ and a
read-only operation labeling $R(e): O \to \mathbb{B}$.

\begin{example}

  TODO MATCHING SCHEME EXAMPLE

\end{example}

The history $H(e,M,R)$ of an execution $e$ under matching scheme $\tup{M,R}$ is
the tuple $\tup{O,<,c,f,M(e),R(e)}$ where
\begin{itemize}

  \item $O$ are the operations of $e$,

  \item $o_1 < o_2$ if{f} operation $o_1$ returns before $o_2$ is called in $e$,

  \item $c(o)$ if{f} operation $o$ returns in $e$, and

  \item $f(o)$ is the name of the method executed by $o$ in $e$.

\end{itemize}
We denote the set $\set{H(e,M,R) : e \in E}$ of histories of an execution set
$E$ by $H(E,M,R)$.

\begin{example}

  TODO HISTORY OF AN EXECUTION

\end{example}

A matching scheme $\tup{M,R}$ is \emph{faithful} to a set $E$ of executions
when $e \in E$ if{f} $e' \in E$ for any two executions $e$ and $e'$ such that
$H(e,M,R) = H(e',M,R)$. A set of executions $E$ is \emph{data
independent}\footnote{TODO NOTE THAT THIS IS NOT THE STANDARD DEFINITION…} when
there exists a faithful matching scheme. By definition, our abstraction of
executions into histories incurs no loss of precision for data-independent
execution sets.

\begin{lemma}

  $H(e,M,R) \in H(E,M,R)$ if and only if $e \in E$, for any faithful matching
  scheme $\tup{M,R}$.

\end{lemma}

In Section~\ref{sec:nature} we demonstrate faithful matching schemes for the
executions of naturally-occurring ADTs, and in Section~\ref{sec:matching} we
demonstrate how to infer faithful matching schemes. Otherwise, for the
remainder of this work, we assume each set of executions comes equipped with a
faithful matching scheme $\tup{M,R}$, and abbreviate $H(E,M,R)$ by $H(E)$.

Two histories $h_1$ and $h_2$ are related by $\to_x$, for $x = \mathrm{o},
\mathrm{c}, \mathrm{p}$, when $h_2$ is obtained from $h_1$ by:
\begin{itemize}

  \item removing an order constraint (o),

  \item making a completed operation pending (c), or

  \item adding a pending operation (p).

\end{itemize}
A set of histories $H$ is \emph{closed} under a relation $\to$ when $h_2 \in H$
whenever $h_1 \to h_2$ and $h_1 \in H$. A fundamental property of
implementations is that their histories are closed under weakening via less
ordering, fewer operations completed, and additional pending
operations~\cite{conf/popl/BouajjaniEEH15}.

\begin{lemma}

  $H(\mathcal{I})$ is closed under $\to_\mathrm{o}$, $\to_\mathrm{p}$, 
  $\to_\mathrm{c}$.

\end{lemma}
