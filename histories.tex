%!TEX root = draft.tex
\section{Implementations \& Histories}
\label{sec:histories}

Abstract data types (ADT) implementations provide methods which can be invoked
concurrently by threads of client programs. We capture the possible histories of
call and return actions in the executions of client programs as
partially-ordered method invocations. While many ADTs are “atomic”
(e.g.,~queues, locks) in the sense that their ADTs are specified as sets of
sequential histories, many (e.g.,~rendezvous synchronizers, barriers) are
non-atomic~\cite{conf/podc/HemedR14}. While much of the following development
could be simplified for atomic ADTs by considering invocation sequences rather
than concurrent invocation histories, we maintain generality so that our results
apply to non-atomic ADTs as well.

To this end, we fix an arbitrary infinite set $\mathbb{O}$ of operation
identifiers, and given sets $\mathbb{M}$ and $\mathbb{V}$ of method names and
values. A \emph{call action} binds an operation $o \in \mathbb{O}$ with a
method name $p \in \mathbb{M}$ and argument value $v \in \mathbb{V}$, while a
\emph{return action} binds an operation $o \in \mathbb{O}$ with a return value
$v \in \mathbb{V}$. An \emph{execution} $e$ is a sequence of call and return
actions where
\begin{itemize}

  \item each operation identifier is used in at most one call action, and in at
  most one return action, and

  \item each return action is preceded by a call action with the same operation
  identifier.

\end{itemize}
An \emph{implementation} $\mathcal{I}$ is a prefix-closed set of executions
which is additionally closed under
\begin{itemize}

  \item appending call actions (of fresh operations),

  \item permuting call actions backward, and

  \item permuting return actions forward.

\end{itemize}
These conditions capture the environment in which implementations execute:
calls are always enabled in their client programs, and thread schedulers may
induce arbitrary delay between implementation code and the associated call and
return actions~\cite{conf/popl/BouajjaniEEH15}.

\begin{example}
  \label{ex:executions}

  Consider the following three executions of a single-value register object
  with read and write methods:
  \begin{center}
    \hfill
    \begin{minipage}{0.3\linewidth}
      $e_1$
      \begin{verbatim}
1: call write(a)
1: return
2: call write(b)
2: return
3: call read
3: return => a
      \end{verbatim}
    \end{minipage}
    \hfill
    \begin{minipage}{0.3\linewidth}
      $e_2$
      \begin{verbatim}
1: call write(a)
2: call write(b)
1: return
2: return
3: call read
3: return => a
      \end{verbatim}
    \end{minipage}
    \hfill
    \begin{minipage}{0.3\linewidth}
      $e_3$
      \begin{verbatim}
1: call write(a)
1: return
2: call write(b)
3: call read
2: return
3: return => a
      \end{verbatim}
    \end{minipage}
    \hfill
  \end{center}
  Operation identifiers precede actions. Executions~$e_2$ and~$e_3$ are obtained
  from $e_1$, respectively, by permuting the return actions of Operations~$1$
  and~$2$ and the call actions of operations $2$ and $3$. Thus while the
  operations of Execution~$e_1$ are sequential, each following the previous in
  time, those of $e_2$ and $e_3$ overlap. While $e_1$ should not be admitted by
  an atomic register, since the read of Operation~$3$ does not return the
  most-recently-written value, both $e_2$ and $e_3$ should be admitted, since
  Operation~$3$ returns the most-recently-written value in some linearization of
  the overlapping operations.

\end{example}

Histories abstract executions, retaining method
names yet losing exact argument and return values, and retaining the relative
order of operations, yet losing the exact sequence of call and return actions.
Formally, a \emph{history} is a tuple $h = \tup{O,<,c,f,m,r}$ where
\begin{itemize}

  \item $O \subseteq \mathbb{O}$ is a set of \emph{operations},

  \item $<$ is a \emph{happens-before} interval order\footnote{An interval
  order~\cite{books/Fishburn85} is a partial order whose elements can be mapped
  to integral intervals preserving the order relation, or equivalently, a
  partial order for which $w < x$ and $y < z$ implies $w < z$ or $y < x$.} on
  $O$,

  \item $c: O \to \mathbb{B}$ labels operations as \emph{completed}, or not,

  \item $f: O \to \mathbb{M}$ labels operations with method names,

  \item $m: O \rightharpoonup O$ is a partial \emph{matching} function, and

  \item $r: O \to \mathbb{B}$ labels operations as \emph{read-only}, or not.

\end{itemize}
The relation $<$ is an interval order because executions’ call and return
actions are totally ordered~\cite{conf/popl/BouajjaniEEH15}. Non-completed
operations are \emph{pending}, and are maximal in the happens-before order.

Two operations $o_1, o_2 \in O$ are \emph{identical} when
\begin{itemize}

  \item they have the same labels: $c(o_1) = c(o_2)$, $f(o_1)= f(o_2)$, and
  $r(o_1) = r(o_2)$, and

  \item they have the same matching: either $m(o_1) = m(o_2)$, or both $m(o_1)$
  and $m(o_2)$ are undefined.

\end{itemize}
The \emph{frequency} of an operation $o$ is the number of operations identical
to $o$, denoted $\freq(o)$. We say that $o$ has \emph{duplicates} when
$\freq(o) > 1$. An operation $o \in \img(m)$ in the image of $m$ is a
\emph{match target}, and the inverse set $m^{-1}(o)$ of a target is a
\emph{match}. We assume $o \in m^{-1}(o)$ for every $o$. A operation $o$ is \emph{unmatched}
when $o$ is completed and $m(o)$ is undefined.

The \emph{width} of a history is the maximum number of its operations which are
mutually unordered, and the \emph{width} of a history set is the maximum width
of its elements. Width-$1$ histories are \emph{sequential}. The \emph{weight}
of a history is the maximum among operation frequencies and the number of
matches. Since the operation identifiers of a history have no intrinsic
meaning, we consider equality between histories up to renaming of operation
identifiers.

\begin{example}
  \label{ex:histories}

  We draw histories by writing one operation per line, starting with its
  operation identifier and match target, followed by its method label,
  possibly a read-only marker, and its happens-before interval. For instance,
  in the history
\begin{verbatim}
    [1:1] write(a)        #
    [2:1] read => a (RO)    #
\end{verbatim}
  Operation~$1$ is a write operation which matches itself, and precedes a
  read-only read operation which also targets Operation~$1$. Note that we
  label method argument and return values for illustrative purposes only;
  histories do not keep them. In the following history, Operations~$1$ and~$2$
  execute concurrently
\begin{verbatim}
    [1:1] write(a)        #
    [2:1] read => a (RO)  #
    [3:X] read => b (RO)    #
    [4:_] read*     (RO)      #
\end{verbatim}
  Operation~$3$ is unmatched, indicated by the {\tt X}, and Operation~$4$ is
  pending, indicated by the {\tt *} on its label.

\end{example}

Our abstraction of executions relies on correlating the operations associated
with the same values in an execution via the partial matching function of a
history. We construct partial matching functions systematically. A
\emph{matching scheme} $\tup{M,R}$ associates to each execution $e$ with
operations $O$ a partial matching function $M(e): O \rightharpoonup O$ and a
read-only operation labeling $R(e): O \to \mathbb{B}$.

\begin{example}
  \label{ex:matching}

  Consider the following matching scheme for the read and write operations of
  a single-value register object:
  \begin{itemize}

    \item write$(v)$ operations match themselves, and

    \item read $\Rightarrow v$ operations are read-only, and match themselves
    when they return $v = -$, or a write$(v)$ operation, if one
    exists, and otherwise have no match,

  \end{itemize}
  which is well defined when each value $v \in \mathbb{V}$ appears as the
  argument of at most one write operation in any execution. This matching
  scheme corresponds to the matching functions in the histories of
  Example~\ref{ex:histories}.

\end{example}

The history $H(e,M,R)$ of an execution $e$ under matching scheme $\tup{M,R}$ is
the tuple $\tup{O,<,c,f,M(e),R(e)}$ where
\begin{itemize}

  \item $O$ are the operations of $e$,

  \item $o_1 < o_2$ if{f} operation $o_1$ returns before $o_2$ is called in $e$,

  \item $c(o)$ if{f} operation $o$ returns in $e$, and

  \item $f(o)$ is the name of the method executed by $o$ in $e$.

\end{itemize}
We denote the set $\set{H(e,M,R) : e \in E}$ of histories of an execution set
$E$ by $H(E,M,R)$. When the matching scheme $\tup{M,E}$ is clear from the
context, we abbreviate $H(e,M,R)$ and $H(E,M,R)$ by $H(e)$ and $H(E)$.

\begin{example}
  \label{ex:histories-of-executions}

  The histories of the executions of Example~\ref{ex:executions} according to
  the matching scheme of Example~\ref{ex:matching} are $H(e_1)$:
\begin{verbatim}
    [1:1] write(a)        #
    [2:2] write(b)          #
    [3:1] read => a (RO)      #
\end{verbatim}
  in which all three operations are sequential, $H(e_2)$:
\begin{verbatim}
    [1:1] write(a)        #
    [2:2] write(b)        #
    [3:1] read => a (RO)    #
\end{verbatim}
  in which the first two operations overlap, and $H(e_3)$:
\begin{verbatim}
    [1:1] write(a)        #
    [2:2] write(b)          #
    [3:1] read => a (RO)    #
\end{verbatim}
  in which the last two operations overlap.

\end{example}

A matching scheme $\tup{M,R}$ is \emph{faithful} to a set $E$ of executions
when $e \in E$ if{f} $e' \in E$ for any two executions $e$ and $e'$ such that
$H(e,M,R) = H(e',M,R)$. A set of executions $E$ is \emph{data
independent}\footnote{Our definition of \emph{data independence} formalizes an
existing informal notion, which stipulates that the implementation generating a
set of executions does not predicate its actions on the data values passed as
method arguments.} when there exists a faithful matching scheme. By definition,
our abstraction of executions into histories incurs no loss of precision for
data-independent execution sets.

\begin{lemma}

  $H(e,M,R) \in H(E,M,R)$ if and only if $e \in E$, for any faithful matching
  scheme $\tup{M,R}$.

\end{lemma}

In Section~\ref{sec:nature} we demonstrate faithful matching schemes for the
executions of naturally-occurring ADTs, and in Section~\ref{sec:matching} we
demonstrate how to infer faithful matching schemes. Otherwise, for the
remainder of this work, we assume each set of executions comes equipped with a
faithful matching scheme $\tup{M,R}$.

Two histories $h_1$ and $h_2$ are related by $\to_x$, for $x = \mathrm{o},
\mathrm{c}, \mathrm{p}$, when $h_2$ is obtained from $h_1$ by:
\begin{itemize}

  \item unordering a pair of ordered operations (o),

  \item making a completed operation pending (c), or

  \item adding a pending operation (p).

\end{itemize}
A set of histories $H$ is \emph{closed} under a relation $\to$ when $h_2 \in H$
whenever $h_1 \to h_2$ and $h_1 \in H$. A fundamental property of
implementations is that their histories are closed under weakening via less
ordering, fewer operations completed, and additional pending
operations~\cite{conf/popl/BouajjaniEEH15}.

\begin{example}

  By un-ordering the first two operations of the history
\begin{verbatim}
    [1:1] write(a)        #
    [2:2] write(b)          #
    [3:2] read => b (RO)      #
\end{verbatim}
  we derive the $\to_\mathrm{o}$-related history
\begin{verbatim}
    [1:1] write(a)        #
    [2:2] write(b)        #
    [3:2] read => b (RO)    #
\end{verbatim}
  from which we can derive the $\to_\mathrm{c}$-related history
\begin{verbatim}
    [1:1] write(a)        #
    [2:2] write(b)        #
    [3:_] read* (RO)        #
\end{verbatim}
  by making Operation~$3$ pending, and from which we can derive the
  $\to_\mathrm{p}$-related history
\begin{verbatim}
    [1:1] write(a)        #
    [2:2] write(b)        #
    [3:_] read* (RO)        #
    [4:4] write(c)*         #
\end{verbatim}
  by adding an additional pending operation.

\end{example}

As these weakening operations align with the environment-capturing closure
properties on executions, the set of histories of an implementation is also
closed.

\begin{lemma}

  $H(\mathcal{I})$ is closed under $\to_\mathrm{o}$, $\to_\mathrm{p}$,
  $\to_\mathrm{c}$.

\end{lemma}
