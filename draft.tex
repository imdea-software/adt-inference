\documentclass[10pt,preprint,nocopyrightspace,natbib,authoryear]{sigplanconf}

\usepackage[utf8]{inputenc}
\usepackage{microtype}
\usepackage{hyperref}
\usepackage{amsmath}
\usepackage{amssymb}
\usepackage{amsthm}

\newtheorem{theorem}{Theorem}
\newtheorem{lemma}{Lemma}
\newtheorem{example}{Example}
\DeclareMathOperator{\dom}{dom}
\DeclareMathOperator{\rng}{rng}
\DeclareMathOperator{\weight}{weight}

\newcommand{\set}[1]{{\{{#1}\}}}
\newcommand{\tup}[1]{{\left\langle{#1}\right\rangle}}

\begin{document}

\special{papersize=8.5in,11in}
\setlength{\pdfpageheight}{\paperheight}
\setlength{\pdfpagewidth}{\paperwidth}

\conferenceinfo{CONF 'yy}{Month d--d, 20yy, City, ST, Country} 
\copyrightyear{20yy} 
\copyrightdata{978-1-nnnn-nnnn-n/yy/mm} 
\doi{nnnnnnn.nnnnnnn}

% Uncomment one of the following two, if you are not going for the 
% traditional copyright transfer agreement.

%\exclusivelicense                % ACM gets exclusive license to publish, 
                                  % you retain copyright

%\permissiontopublish             % ACM gets nonexclusive license to publish
                                  % (paid open-access papers, 
                                  % short abstracts)

% \titlebanner{banner above paper title}        % These are ignored unless
% \preprintfooter{short description of paper}   % 'preprint' option specified.

\title{Abstract Data Type Inference}
% \subtitle{}

\authorinfo{Michael Emmi}
           {IMDEA Software Institute}
           {michael.emmi@imdea.org}
\authorinfo{Constantin Enea\and Jad Hamza}
           {Université Paris Diderot}
           {\{cenea,jhamza\}@univ-paris-diderot.fr}

\maketitle

\begin{abstract}

  Checking that efficient implementations of concurrent objects adhere to their
  abstract data types (ADTs) can be made tractable when ADTs are given by
  symbolic, logical representations. Obtaining these symbolic representations,
  however, is a complex task requiring rare expertise. In practice, concurrent
  objects are specified indirectly by imperative reference implementations.

  In this work we demonstrate that effective symbolic ADT representations can
  be automatically generated from the executions of reference implementations.
  Our approach exploits key features of naturally-occurring ADTs: violations
  can be decomposed into a small set of representative patterns, and these
  patterns manifest in executions with few operations. By identifying certain
  algebraic properties of naturally-occurring ADTs, and exhaustively sampling
  executions up to a small number of operations, we generate concise symbolic
  ADT representations which are complete in practice, enabling the application
  of efficient symbolic verification algorithms without the burden of manual
  specification.

\end{abstract}

\category{F.3.1}{Specifying and Verifying and Reasoning about Programs}{Mechanical verification}

\terms
Reliability, Verification

\keywords
Concurrency; Refinement; Linearizability

\input intro
\input inference
\input algebra
\input algorithm
\input matching
\input nature
\input empirical
\input related

% \appendix

% \acks …

% \balancecolumns

\bibliographystyle{abbrvnat}
\bibliography{violin}

% % The bibliography should be embedded for final submission.
%
% \begin{thebibliography}{}
% \softraggedright
%
% \bibitem[Smith et~al.(2009)Smith, Jones]{smith02}
% P. Q. Smith, and X. Y. Jones. ...reference text...
%
% \end{thebibliography}

\end{document}
