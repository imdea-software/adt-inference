\section{Introduction}
\label{sec:intro}

Effective scalable reasoning about nontrivial software implementations generally
requires considering each software module separately, in isolation, using
abstract specifications for other modules. When modules are objects whose
methods may be called concurrently, their behavior is typically understood in
terms of invocation sequences of abstract data types (ADTs): an execution with
overlapping method invocations is considered valid when those invocations can be
\emph{linearized} into a sequence admitted by the
ADT~\cite{journals/toplas/HerlihyW90}. For example, consider the execution
history depicted in Figure~\ref{fig:clients} in which the add operations
numbered $2$ and $3$ overlap with each other, and, respectively, with operations
$1$ and $4$. This execution is valid with respect to the atomic queue ADT
because among the five possible ways to linearize its six operations, the
sequence~$1$, $3$, $2$, $4$, $5$, $6$ is admitted. ADT specifications thus
decouple reasoning about object implementations from their clients’ invocations:
\begin{itemize}

  \item Is there a valid linearization for each implementation execution?

  \item Does every valid linearization preserve client invariants?

\end{itemize}
The former question depends only on a given object’s implementation, and the
latter only on a given object’s clients.

\begin{figure}[t]
  \begin{minipage}{0.43\linewidth}
    \begin{verbatim}
1: add(a)       #
2: add(b)       ###
3: add(c)         ###
4: add(d)           #
5: remove => a        #
6: remove => c          #
    \end{verbatim}
  \end{minipage}
  \hfill
  \begin{minipage}{0.55\linewidth}
    \begin{verbatim}
assume w < x
(  (q.add(<1,w>); q.add(<1,x>))
|| (q.add(<2,w>); q.add(<2,x>)) )
i, y := q.remove()
j, z := q.remove()
assert i == j ==> y < z
    \end{verbatim}
  \end{minipage}
  \caption{An execution history with six numbered operations (left),
    and a parallel program invoking six operations (right). The “\#” symbols
    depict the time intervals spanned by operations horizontally.}
  \label{fig:clients}
\end{figure}

\begin{example}

  Consider the parallel program in Figure~\ref{fig:clients} invoking the add and
  remove methods of an atomic queue implementation, adding increasing integer
  values $w$ and $x$ tagged with the integers $\set{1,2}$ indicating on which
  parallel branch each add occurs. Intuitively this program is correct since
  values with the same tag are added in increasing order, and, crucially, the
  values of the queue ADT are removed in the same order in which they are added.
  Among the six possible ways to linearize these operations, the comparison {\tt i ==
  j} of tags only holds for those two beginning with
  \begin{quote}
    \verb|q.add(<1,w>); q.add(<1,x>) | and \verb| q.add(<2,w>); q.add(<2,x>)|
  \end{quote}
  Since the queue ADT dictates that elements are removed in the order added, we
  conclude that $w$ and $x$ are removed in order when {\tt i == j}, and thus {\tt
  y < z} holds when {\tt i == j} holds.

\end{example}

Although formal ADT specifications are indispensable for scalable program
reasoning, formal-specification writing is a burden for which few programmers
posses the required combination of expertise and willingness to overcome.
Typically programmers write simple ADT \emph{reference implementations},
e.g.,~whose methods are synchronized via a global lock, and refine them with
more efficient fine-grained implementations, e.g.,~reducing synchronization
bottlenecks using specialized hardware instructions such as atomic compare and
swap (CAS).

Our goal in this work is thus the automated generation of formal ADT
specifications, derived from reference implementations, which are suitable for
automated reasoning. In particular, we aim to generate \emph{symbolic
representations} of valid invocation sequences for ADTs which are given
implicitly by reference implementations. We target declarative symbolic
representations rather than imperative state-based representations in order to
harness efficient symbolic reasoning algorithms: rather than enumerating
linearizations explicitly, and checking their validity one by one, a symbolic
reasoning engine may simultaneously rule out many possible linearizations.
Previous work demonstrates that such symbolic reasoning can increase
efficiency by orders of magnitude~\cite{conf/pldi/EmmiEH15}.

In this work we demonstrate that effective symbolic ADT representations can be
generated automatically from the executions of reference implementations,
enabling the application of efficient symbolic reasoning algorithms
without the burden of writing formal specifications manually. Our approach
exploits two key features of concurrent-object ADTs: that violations of each
ADT can be decomposed into a small set of representative patterns, and that
these patterns manifest in executions with few operations. The first feature
allows us to represent symbolic ADTs finitely, as exclusions of violation
patterns. The second allows us to extract violation patterns from finite
enumerations of executions.

The fundamental challenge is in identifying the algebraic properties of ADTs
which characterize an infinite set of violating executions with a
finite set of patterns. This characterization is non-trivial since an execution
with more operations than a given violating execution is not necessarily a
violation itself. For instance, an execution which contains only a single
pop operation returning the value $1$ is a violation to the atomic stack
ADT, whereas an execution containing an additional push$(1)$ operation,
overlapping in time with the pop, is not. Further complication arises
from the infinite set of possible data values, i.e.,~method argument and return
values. Our patterns must be sensitive to the \emph{relation} among data values
without being sensitive to the data values themselves. For instance, a
sequential execution in which $1$ and $2$ are pushed and subsequently
popped in the same order violates the atomic stack ADT. Yet, while replacing
both values $1$ and $2$ with the value $1$ results in a
\emph{valid} stack execution, replacing them with $3$ and $4$ results
in a violation.

Our algebraic insight is based on grouping the operations of an execution into
\emph{matchings}. Intuitively, operations which refer to the same instances of
values belong to the same matching. For example, a pop operation
returning the value $1$ matches a preceding push$(1)$ operation. By
comparing executions by the characteristics of their matchings, rather than the
actual data values they use, we capture the relation among data values
independently of the data values themselves. Furthermore, the notion of
matchings provides a key algebraic property of ADTs: the executions of
naturally-occurring concurrent object ADTs are closed under the removal of
matchings. For example, any execution of the atomic stack ADT using values
$1$, $2$, and $3$ would remain a valid execution were all operations
using the value $2$ deleted. Conversely, any execution which extends a
violating execution with additional matchings is itself a violation. This
property, along with analogous algebraic properties concerning operation order
and completion, allow us to compare executions via a violation-preserving
embedding relation. This relation is a well-founded partial order on
executions, and thus allows us to characterize the infinite set of violations
to an ADT with a finite basis set, ultimately leading to a finite symbolic
representation.

Computing the basis sets of ADT violation patterns is a challenging problem,
requiring the computation of global properties of an infinite number of
executions — analogously to the inference of inductive invariants. Exploiting a
hypothesis that violation patterns manifest in executions with few operations,
we propose an under-approximating algorithm which extracts the patterns observed
in all violating executions up to a given number of operations. In theory, for
an arbitrary ADT, this is clearly incomplete: any violation which only surfaces
with a greater number of operations would not be captured, thus resulting in a
symbolic ADT representation which can fail to identify violations — though still
guarantees never to classify a valid execution as a violation, and is thus sound
for program reasoning. Empirically, however, we demonstrate that our hypothesis
holds: the patterns emerging from executions with few operations are complete in
characterizing all violations of naturally-occurring concurrent object ADTs,
thus allowing us to compute complete symbolic ADT representations in practice.

Although our approach does require annotating the operations of concurrent
object executions with a matching relation, and we demonstrate that these
relations are easy to provide for naturally-occurring ADTs, we also outline an
automatic means for computing such \emph{matching schemes}. Again by sampling
executions, we leverage automated symbolic reasoning engines to synthesize
matching schemes for which given implementations are closed under the removal
of matches. This further lowers the burden of automated verification. Rather
than providing formal ADT specifications, or even matching schemes, users need
only provide the predicates relevant in the logic of matching schemes, and we
could automatically compute effective matching schemes, and ultimately,
effective symbolic ADT representations.

In summary, the contributions and outline of this work are:
\begin{itemize}

  \item An abstract notion of execution histories based on groups of matching
  operations (§\ref{sec:histories}).

  \item The statement of the symbolic ADT inference problem (§\ref{sec:inference}).

  \item Identification of the algebraic properties allowing a finite
  characterization of infinite ADT violation sets (§\ref{sec:patterns}).

  \item The symbolic representation of ADT violations (§\ref{sec:formula}).

  \item The computability of symbolic ADT representations (§\ref{sec:algorithm}).

  \item An algorithm to infer the matching schemes required for our algebraic
  characterization of ADTs (§\ref{sec:matching}).

  \item An empirical study validating that naturally-occurring ADTs satisfy the
  properties required for completeness of our inference algorithm, and that our
  algorithm computes precise symbolic representations thereof
  (§\ref{sec:nature}).

\end{itemize}
We conclude by discussing limitations
(§\ref{sec:discussion}) and related work (§\ref{sec:related}).

To the best of our knowledge, this work is the first to suggest the automatic
generation of ADT specifications. By removing the burden of writing formal ADT
specifications manually, this work broadens the scope of modular program
reasoning using efficient symbolic algorithms to newly-designed ADTs, apart from
those few traditionally studied in the literature.
