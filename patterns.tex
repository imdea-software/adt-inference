%!TEX root = draft.tex
\section{Finite ADT Representations}
\label{sec:patterns}

In this section we demonstrate that naturally-occurring ADTs can be precisely
represented by finite sets of histories, despite the fact that these ADTs admit
infinite sets of histories. This results relies on the identification of certain
algebraic properties of the sets of histories admitted by ADTs. In particular,
we find that sets of histories admitted by ADTs are closed under the removal
of certain operations, and that these sets adhere to a well-founded ordering
under the relation induced by such removals.

In addition to the relations $\to_\mathrm{o}$, $\to_\mathrm{p}$, and
$\to_\mathrm{c}$ of Section~\ref{sec:histories} under which all implementation
history sets are closed, the histories of ADT implementations we consider in
this work are also closed under additional relations which remove read-only
operations, unmatched operations, entire matches, and duplicate operations. The
relations $\to_\mathrm{r}$, $\to_\mathrm{u}$, $\to_\mathrm{m}$ and
$\to_\mathrm{d}$ relate two histories $h_1$ and $h_2$ when $h_2$ is obtained
from $h_1$ by
\begin{itemize}

  \item removing a read-only operation (r),

  \item removing an unmatched operation (u),

  \item removing a match (m), or

  \item removing a duplicate operation (d).

\end{itemize}
We say an ADT whose histories are closed under $\to_\mathrm{r}$,
$\to_\mathrm{u}$, $\to_\mathrm{m}$ and $\to_\mathrm{d}$ is \emph{normal}. In
Section~\ref{sec:nature} we demonstrate that naturally-occurring ADTs are
normal. Defining the relation $\succeq$ as the reflexive and transitive closure
of the above relations,
\begin{align*}
  \mathord{\succeq} = (
    \to_\mathrm{o} \cup \to_\mathrm{p} \cup \to_\mathrm{c} \cup
    \to_\mathrm{r} \cup \to_\mathrm{u} \cup \to_\mathrm{m} \cup \to_\mathrm{d}
  )^\ast
\end{align*}
closure under $\succeq$ follows immediately.

\begin{lemma}
  \label{lem:ADT_closure}

  Normal ADTs are closed under $\succeq$.

\end{lemma}

Besides this closure property, the inverse $\preceq$ relation enjoys a certain notion of
well-foundedness when restricted to bounded-width histories: the set of
$\preceq$-minimal elements of any (potentially infinite) history set is finite.
This property is what enables us to represent infinite sets of invalid ADT
histories with a finite set of minimal examples. This property is captured
formally with wqos: a \emph{well-quasi-ordering (wqo)} $R$ on a set $X$ is a
reflexive, transitive binary relation on $X$ for which in every infinite
sequence $x_0 x_1 \ldots$ of elements from $X$, there exists indices $i < j$
such that $R(x_i,x_j)$.

\begin{example}

  Consider the infinite history sequence $h_1 h_2 \ldots$ where each $h_i$
  contains $2i$ operations $o_1, o_1', \ldots, o_i, o_i'$ where each $o_j$ is a
  completed {\tt push} operation matching itself, and each $o_j'$ is a pending
  {\tt pop} operation with undefined matching. Because each successive $h_i$
  has both more matches and more pending operations, no two histories of the
  sequence are related by $\preceq$. Thus $\preceq$ is not a wqo.

\end{example}

This example demonstrates that $\preceq$ is not a wqo by allowing each history
$h_i$ of the infinite sequence to contain more and more pending operations in
order to ensure that $h_j \not\preceq h_i$ for every $j < i$. Curbing this
ability by limiting the maximum amount of pending operations per history makes
$\preceq$ a wqo. Although limiting to $k$ pending operations essentially limits
us to width-$k$ histories, of executions with at most $k$ operations parallel at
any moment, e.g.,~of programs with at most $k$ threads, this restriction comes
at no loss of completeness when considering only the histories of bounded-width
ADTs, which is the subject of the remainder of this section.

\begin{lemma}
  \label{lem:wqo}

  $\preceq$ is a wqo on bounded-width histories.\footnote{The proof of
  Lemma~\ref{lem:wqo} appears in Appendix~\ref{sec:proof}.}

\end{lemma}

For the remainder of this section, we fix an ADT $A$, and let $B$ be its bounded
complement. When $A$ has bounded width, so does $B$, and thus $\preceq$ is a wqo
on $B$. Furthermore, when $A$ is normal, it is closed under $\succeq$, and thus $B$
is closed\footnote{Actually, $B$ is closed under $\preceq$ when restricted to
width-bounded histories, i.e., if $h_1 \preceq h_2$, $h_1 \in B$, and $h_2$ is
width-bounded, then $h_2\in B$.} under $\preceq$. Closure under a relation
satisfying Lemma~\ref{lem:wqo} implies representation by a finite set. Formally,
we say a set $X$ is \emph{finitely representable} if there exists a finite set
$Y$ and a relation $R \subseteq Y \times X$ such that $X = \set{ x : \exists
y\in Y.\ R(y,x) }$. In our case, we obtain a finite set from which exactly the
elements of $B$ are related by $\preceq$.

\begin{lemma}

  $B$ is finitely representable if $A$ is normal.

\end{lemma}

%Computing $B$ thus amounts to computing its finite representation. In general,
%this is achieved by enumerating the elements of $B$ while maintaining the
%$\preceq$-minimal elements, and recognizing a condition under which all
%elements of $B$ are related to the current set of
%minimals~\cite{conf/lics/AbdullaCJT96, journals/tcs/FinkelS01}. In our case, we
%recognize a condition on the weight-increasing enumeration of $B$ which
%guarantees coverage by the current set of minimals. Formally, for all $i \in
%\mathbb{N}$, we define
%\begin{align*}
%  B_i &= \set{ h \in B : \weight(h) \le i } \\
%  B_i' &= \set{ h' \in B :
%    \exists h \in B_i.\ h \preceq h' \text{ and } \weight(h') \le i+1
%  }
%\end{align*}
%as, respectively, the histories of $B$ with at most $i$ matches and duplicates,
%and those derived from $B_i$ with at most $i\!+\!1$ matches and duplicates. We
%say that $A$ is \emph{predictable} if $B_i = B$ whenever $B_i' = B_{i+1}$. In
%Section~\ref{sec:nature} we demonstrate that naturally-occurring ADTs are
%predictable. Since the number of weight-$i$ histories is finite\footnote{As
%noted in Section~\ref{sec:histories}, we consider equality between histories up
%to renaming of operation identifiers.} for each $i \in \mathbb{N}$, each $B_i$
%is computable.
%
%\begin{lemma}
%
%  $B_i$ is computable, for all $i \in \mathbb{N}$, if $A$ is normal.
%
%\end{lemma}
%
%Predictability ensures that once $B_{i+1}$ does not alter the set of computed
%$\preceq$-minimals, with respect to those of $B_i$, then the minimals computed
%for $B_i$ are exactly those of $B$.
%
%\begin{lemma}
%
%  $B$ is computable if $A$ is normal and predictable.
%
%\end{lemma}
%
%Note that for non-predictable ADTs, computing $B_i$, for any $i \in
%\mathbb{N}$, under-approximates $B$. In practice, computing $B_i$ can be used
%to identify all violations which surface in histories with $i$ matches,
%overlooking violations which only surface with more than $i$ matches. The
%resulting symbolic ADT representation would still be sound in only identifying
%\emph{actual} violations, yet incomplete in identifying \emph{all} violations.

\begin{example}
  \label{ex:patterns}

  The following four histories generate the complement of the atomic
  single-value register ADT, witnessing either an unmatched read operation:
\begin{verbatim}
  [1:X] read => 1 (RO)  #
\end{verbatim}
  a read of an uninitialized register occurring after some write:
\begin{verbatim}
  [1:1] write(1)        #
  [2:2] read => - (RO)    #
\end{verbatim}
  a read which happens before its matching write operation:
\begin{verbatim}
  [1:2] read => 1 (RO)  #
  [2:2] write(1)          #
\end{verbatim}
  and a read matching a write which is not the most recent:
\begin{verbatim}
  [1:1] write(1)        #
  [2:2] write(2)          #
  [3:1] read => 1 (RO)      #
\end{verbatim}
  Every sequential history not admitted by the atomic register ADT embeds at
  least one of these four histories.

\end{example}
