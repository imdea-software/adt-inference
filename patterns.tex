%!TEX root = draft.tex
\section{Computing ADT Complements}
\label{sec:patterns}

TODO MOTIVATE THIS SECTION; ALSO, RETHINK THE TITLE OF THIS SECTION

In addition to the relations $\to_\mathrm{o}$, $\to_\mathrm{p}$, and
$\to_\mathrm{c}$ of Section~\ref{sec:histories} under which all ADTs are
closed, the ADTs we consider in this work are also closed under additional
relations which remove read-only operations, unmatched operations, entire
matches, and duplicate operations. The relations $\to_\mathrm{r}$,
$\to_\mathrm{u}$, $\to_\mathrm{m}$ and $\to_\mathrm{d}$ relate two histories
$h_1$ and $h_2$ when $h_2$ is obtained from $h_1$ by
\begin{itemize}

  \item removing a read-only operation (r),

  \item removing an unmatched operation (u),

  \item removing a match (m), or

  \item removing a duplicate operation (d).

\end{itemize}
We say an ADT closed under $\to_\mathrm{r}$, $\to_\mathrm{u}$, $\to_\mathrm{m}$
and $\to_\mathrm{d}$ is \emph{normal}. In Section~\ref{sec:nature} we
demonstrate that naturally-occurring ADTs are normal. Defining the relation
$\succeq$ as the reflexive and transitive closure of the above relations,
\begin{align*}
  \mathord{\succeq} = (
    \to_\mathrm{o} \cup \to_\mathrm{p} \cup \to_\mathrm{c} \cup 
    \to_\mathrm{r} \cup \to_\mathrm{u} \cup \to_\mathrm{m} \cup \to_\mathrm{d}
  )^\ast
\end{align*}
closure under $\succeq$ follows immediately.

\begin{lemma}

  Normal ADTs are closed under $\succeq$.

\end{lemma}

TODO MOTIVATE WQO

A \emph{well-quasi-ordering (wqo)} $R$ on a set $X$ is a reflexive, transitive
binary relation on $X$ for which in every infinite sequence $x_0 x_1 \ldots$ of
elements from $X$, there exists indices $i < j$ such that $R(x_i,x_j)$.

\begin{example}

  Consider the infinite history sequence $h_1 h_2 \ldots$ where each $h_i$
  contains $2i$ operations $o_1, o_1', \ldots, o_i, o_i'$ where each $o_j$ is a
  completed {\tt push} operation matching itself, and each $o_j'$ is a pending
  {\tt pop} operation with undefined matching. Because each successive $h_i$
  has both more matches and more pending operations, no two histories of the
  sequence are related by $\preceq$. Thus $\preceq$ is not a wqo.

\end{example}

The previous example demonstrates that $\preceq$ is not a wqo by allowing each
history $h_i$ of the infinite sequence to contain more and more pending
operations in order to ensure that $h_j \not\preceq h_i$ for every $j < i$.
Curbing this ability by limiting the maximum amount of pending operations per
history makes $\preceq$ a wqo. Although limiting to $k$ pending operations
essentially limits us to width-$k$ histories, of executions with at most $k$
operations parallel at any moment, e.g.,~of programs with at most $k$ threads,
this restriction comes at no loss of completeness when considering only the
histories of bounded-width ADT kernels, which is the subject of the remainder
of this section.

\begin{lemma}

  $\preceq$ is a wqo on width-bounded histories.

\end{lemma}

\begin{proof}

%We prove that $\preceq$ is well-founded, i.e., there is no infinite strictly decreasing sequence $h_1 \succ h_2\succ \ldots$\footnote{We define $\succ$ as usual by $h\succ h'$ iff $h\succeq h'$ and $h\neq h'$.},
%and that $\preceq$ has no infinite antichain, i.e., there is no infinite set $\{h_1,h_2,\ldots\}$ of mutually incomparable elements 
%$h_i\not\preceq h_j$ and $h_j\not\preceq h_i$ when $i\neq j$.
%
%Concerning well-foundedness, note that a given history $h_1$ can be weaken only finitely many times by removing matches, duplicates in a match, or happens-before ordering constraints, or by turning completed operations into pending operations. Since we restrict ourselves to width-bounded histories, $h_1$ can also be weaken only finitely many times by adding pending operations (adding a pending operation increases the width of a history by 1).
%
%TODO PROBLEM WITH PENDING OPS

%To prove that $\preceq$ has no infinite antichain, 
%we use the following definitions.
Let $h_1 h_2\ldots $ be an infinite sequence of histories. We prove that there exists $i<j$ such that $h_i\preceq h_j$.

The \emph{signature} of an operation $o$ in a history $h$ is the pair $\sigma(o)=(c(o),f(o))$.
Then, the \emph{signature} of a match $\mu=\set{ o' \in O : m(o') = o}$ is the pair 
\[
\sigma(\mu)=(\sigma(o),\set{\sigma(o') : m(o')=o, o'\neq o})
\]
containing the signature of the match target and the set of the signatures of the other operations.
The \emph{signature} of a history $h$ is the set 
\[
\sigma(h)=\set{\sigma(\mu): \mu\mbox{ a match of $h$}}\cup\set{\sigma(o): o\mbox{ unmatched}}.
\]
of match signatures and signatures of unmatched operations.

%Assume by contradiction that $\preceq$ has an infinite antichain.
Since the set of history signatures is bounded (because the set of methods is bounded)
and the set of pending operations is also bounded (since we consider only bounded width histories),
the sequence $h_1 h_2\ldots $ contains an infinite sequence of histories $h_1' h_2' \ldots$ that have 
the same signature and the same number of pending operations with the same signature.

%The \emph{vector} of a match $\mu$ is the pair
%\[
%\nu(\mu)=(\sigma(o),\mset{\sigma(o') : m(o')=o, o'\neq o})
%\]
%containing the signature of the match target and the multiset of the signatures of the other operations.
%Also, 
The \emph{vector} of a history $h$ is the multiset 
\[
\nu(h)=\mset{\sigma(\mu): \mu\mbox{ a match of $h$}}
\]
of match signatures in $h$.

Given a finite alphabet $\Sigma$, a domain $\mathbb{D}$ enhanced with a wqo $\leq$, and two functions 
$\alpha,\beta:\Sigma\rightarrow \mathbb{D}$, let $\sqsubseteq$ be an order relation defined by
\[
\alpha\sqsubseteq\beta\mbox{ iff for every $a\in \Sigma$, $\alpha(a)\leq\beta(a)$}.
\]
Note that $\sqsubseteq$ is a wqo.

The relation $\sqsubseteq$ on history vectors (mappings from the alphabet $\Sigma$ of match signatures to $\mathbb{N}$), 
where $\leq$ is the order relation on natural numbers, is a wqo.
Hence, the set $H$ contains an infinite sequence $h_1',h_2',\ldots$ such that $\nu(h_1')\sqsubseteq \nu(h_2')\sqsubseteq \ldots$.

A \emph{nested vector} of a history $h$ is the nested multiset
\[
\nu^2(h)=\mset{\mset{\nu(\mu): \sigma(\mu)=\sigma} : \sigma\mbox{ the signature of a match in $h$}}.
\]


Again using the fact that $\leq$ is a wqo, the sequence $h_1',h_2',\ldots$ contains a history $h'_k$ and an infinite set of histories 
$\set{h_1'',h_2'',\ldots}$ such that for every $i$ and every match $\mu$ of $h_i''$,
$\nu(\mu_k)\leq \nu(\mu)$, for some match $\mu_k$ of $h_k$ such that $\sigma(\mu_k)=\sigma(\mu)$.

\end{proof}

For the remainder of this section, we fix an ADT $A$, and let $B$ be the
complement of $\ker A$. When $\ker A$ has bounded width, so does $B$, and thus
$\preceq$ is a wqo on $B$. Furthermore, when $A$ normal, it is closed under
$\succeq$, and thus $B$ is closed under $\preceq$. Closure under a wqo implies
representation by a finite set. Formally, we say a set $X$ is \emph{finitely
representable} if there exists a finite set $Y$ and a relation $R \subseteq Y
\times X$ such that $X = \set{ x : \exists y\in Y.\ R(y,x) }$. In our case, we
obtain a finite set from which exactly the elements of $B$ are related by
$\preceq$.

\begin{lemma}

  $B$ is finitely representable if $A$ is normal.

\end{lemma}

Computing $B$ thus amounts to computing its finite representation. In general,
this is achieved by enumerating the elements of $B$ while maintaining the
$\preceq$-minimal elements, and recognizing a condition under which all
elements of $B$ are related to the current set of
minimals~\cite{conf/lics/AbdullaCJT96, journals/tcs/FinkelS01}. In our case, we
recognize a condition on the weight-increasing enumeration of $B$ which
guarantees coverage by the current set of minimals. Formally, for all $i \in
\mathbb{N}$, we define
\begin{align*}
  B_i &= \set{ h \in B : \weight(h) \le i } \\
  B_i' &= \set{ h' \in B :
    \exists h \in B_i.\ h \preceq h' \text{ and } \weight(h') \le i+1
  }
\end{align*}
as, respectively, the histories of $B$ with at most $i$ matches and duplicates,
and those derived from $B_i$ with at most $i\!+\!1$ matches and duplicates. We
say that $A$ is \emph{predictable} if $B_i = B$ whenever $B_i' = B_{i+1}$. In
Section~\ref{sec:nature} we demonstrate that naturally-occurring ADTs are
predictable. Since the number of weight-$i$ histories is finite\footnote{As
noted in Section~\ref{sec:histories}, we consider equality between histories up
to renaming of operation identifiers.} for each $i \in \mathbb{N}$, each $B_i$
is computable.

\begin{lemma}

  $B_i$ is computable, for all $i \in \mathbb{N}$, if $A$ is normal.

\end{lemma}

Predictability ensures that once $B_{i+1}$ does not alter the set of computed
$\preceq$-minimals, with respect to those of $B_i$, then the minimals computed
for $B_i$ are exactly those of $B$.

\begin{lemma}

  $B$ is computable if $A$ is normal and predictable.

\end{lemma}

Note that for non-predictable ADTs, computing $B_i$, for any $i \in
\mathbb{N}$, under-approximates $B$. In practice, computing $B_i$ can be used
to identify all violations which surface in histories with $i$ matches,
overlooking violations which only surface with more than $i$ matches. The
resulting symbolic ADT representation would still be sound in only identifying
\emph{actual} violations, yet incomplete in identifying \emph{all} violations.

\begin{example}

  TODO EXAMPLE OF COMPLETE PATTERNS

  \begin{figure}[t]
    \centering
    \includegraphics[scale=0.2]{figures/register-pattern-1},
    \includegraphics[scale=0.2]{figures/register-pattern-2},
    \includegraphics[scale=0.2]{figures/register-pattern-3}
    \caption{Three histories which generate the complement of the atomic
      register ADT kernel.}
    \label{fig:register-patterns}
  \end{figure}

\end{example}
