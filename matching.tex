\section{Computing Matching Schemes}
\label{sec:matching}

TODO MOTIVATE THIS SECTION, AND ALSO ACCOUNT FOR THE READ-ONLY OPERATION
LABELING TOO.

A \emph{language} $L = \tup{E,P,Q}$ is a set $E$ of executions, along with
finite sequences $P = P_1 P_2 \ldots$ and $Q = Q_1 Q_2 \ldots$ of binary and
ternary predicates $P_i(e,o_1)$ and $Q_i(e,o_1,o_2)$ ranging over the
executions of $E$ and their operations. When $P$ is a $k$-length sequence of
$n$-ary predicates, we write $P(x_1, \ldots, x_n)$ to denote the valuation
sequence
\begin{align*}
  P_1(x_1, \ldots, x_n) \ldots P_k(x_1, \ldots, x_n).
\end{align*}
Given a language $L = \tup{E,P,Q}$, we say a matching scheme $M$ is
\emph{simple} when there exists an $n$-ary Boolean function $G$, for $n =
2\cdot|P|+|Q|$, such that for each execution $e \in E$
\begin{itemize}

  \item $G(P(e,o_1),P(e,o_2),Q(e,o_1,o_2))$ is satisfied for at most one
  operation $o_1$, for any operation $o_2$,

  \item $M(e)(o_2)$ is undefined unless there exists an operation $o_1$
  for which $G(P(e,o_1),P(e,o_2),Q(e,o_1,o_2))$, and

  \item $M(e)(o_2) = o_1$ if{f} $G(P(e,o_1),P(e,o_2),P(e,o_1,o_2))$,

\end{itemize}
where $o_1$ and $o_2$ range over the operations of $e$.

\begin{example}

  We say an execution of read and write methods \emph{writes unique values} if
  the argument value to each write operation is unique. Consider the language
  whose executions write unique values, with the following predicates:
  \begin{align*}
    & \mathsf{write}(e,o) & \text{$o$ is a {\sf write} operation in $e$}, \\
    & \mathsf{read}(e,o) & \text{$o$ is a {\sf read} operation in $e$,} \\
    & \mathsf{eq}(e,o_1,o_2) & \text{$o_1$ and $o_2$ are the same operation, and} \\
    & \mathsf{veq}(e,o_1,o_2) & \text{$o_1$ and $o_2$ read/write the same values.}
  \end{align*}
  Given a valuation $\tup{x_\mathsf{w}, x_\mathsf{r}, y_\mathsf{w},
  y_\mathsf{r}, z_\mathsf{=}, z_\mathsf{v}}$ of the predicates above, we define
  the function $G(x_\mathsf{w}, x_\mathsf{r}, y_\mathsf{w}, y_\mathsf{r},
  z_\mathsf{=}, z_\mathsf{v})$ to be satisfied if and only if:
  \begin{align*}
    ( x_\mathsf{w} \land z_\mathsf{=} )
    \lor ( x_\mathsf{w} \land y_\mathsf{r} \land z_\mathsf{v} )
  \end{align*}
  Intuitively, this defines a simple matching scheme for which write
  operations match themselves, and read operations match the write which wrote
  the value read. In the case such a write exists, it is unique in any
  execution which writes unique values. The match is otherwise undefined.

\end{example}

An implementation $\mathcal{I}$ \emph{adheres} to a language $L = \tup{E,P,Q}$
if $\mathcal{I} \subseteq E$. A match scheme $M$ \emph{normalizes} an
implementation $\mathcal{I}$ when $M$ is faithful to $\mathcal{I}$ and
$H(\mathcal{I},M)$ is normal.

\begin{example}

  TODO SHOW THAT PREVIOUS SIMPLE MATCHING SCHEME NORMALIZES SOMETHING

\end{example}

\begin{definition}

  The \emph{matching scheme inference problem} is to compute a simple matching
  scheme $M$ which normalizes a given implementation $\mathcal{I}$ adhering to
  a given language $L$.

\end{definition}

We automate the computation of matching schemes by considering the function
${\sf g}$ underlying a simple matching scheme as an uninterpreted function in a
logical satisfiability problem, and constrain ${\sf g}$ to normalize the
executions of an implementation. To this end, we fix an implementation
$\mathcal{I}$ and a language $L = \tup{E,P,Q}$ to which $\mathcal{I}$ adheres,
then consider any enumeration $F$ of execution pairs $\tup{e,e'} \in E^2$ such
that
\begin{itemize}

  \item $e \in \mathcal{I}$ and $e' \not\in \mathcal{I}$, and

  \item $e'$ is obtained by deleting operations of $e$.

\end{itemize}
Any such pair of executions can be used to rule out several possibilities,
for instance, that the operations removed from $e$ to obtain $e'$:
\begin{itemize}

  \item do not constitute a match in $e$,

  \item are not all duplicate operations,

  \item are not all read-only operations,

  \item do not constitute multiple matches in $e$,

\end{itemize}
and so on. Ruling out these possibilities is sound since, for example, a
normalized scheme $M$ could not consider those operations a match: otherwise
the history abstraction $H(\mathcal{I},M)$ which includes $H(e,M)$ must also
include $H(e',M)$, being normal, and in particular closed under match removal.
Such an $M$ would thus not be faithful. For simplicity, in what follows we
consider ruling out only the first possibility: that the operations removed
from $e$ are not a match. In principle, the approach extends to rule out all
possibilities.

In what follows, we denote the operations of an execution $e$ by $O_e$.
To consider whether a given pair $o_1, o_2$ of operations of an execution $e$
is a match according to the simple matching scheme based on ${\sf g}$, for
the given language $L$, we define the macro $\textsc{IsMatch}_{e,o_1,o_2}$:
\begin{align*}
  \mathsf{g}\big( P(e,o_1), P(e,o_2), Q(e,o_1,o_2) \big).
\end{align*}
Then a given set $O \subseteq O_e$ constitutes a match according to
${\sf g}$ when all operations $o_2 \in O$ target some operation $o_1 \in O$,
and no other operation $o_2 \in O \setminus O_e$ does. We express this with the
macro $\textsc{EntireMatch}_{e,O}$:
\begin{align*}
  \bigvee_{o_1 \in O}
  \Big(
  \bigwedge_{o_2 \in O} \textsc{IsMatch}_{e,o_1,o_2}
  \land \bigwedge_{o_2 \in O_e \setminus O} \lnot \textsc{IsMatch}_{e,o_1,o_2}
  \Big)
\end{align*}
Finally, given a pair $\tup{e,e'} \in F$, we prohibit the operations $O_e
\setminus O_{e'}$ from constituting a match according to ${\sf g}$, since if
${\sf g}$ normalized $\mathcal{I}$, and $e \in \mathcal{I}$, then $e'$ should
also be in $\mathcal{I}$. We express this exclusion for all pairs of $F$
with the macro $\textsc{Normalizes}_{F}$:
\begin{align*}
  \bigwedge_{\tup{e,e'}\in F} \lnot \textsc{EntireMatch}_{e,(O_e \setminus O_{e'})}
\end{align*}
We thus check satisfiability for the conjunction of non-matches $O_e \setminus
O_{e'}$ over all pairs $\tup{e,e'} \in F$.

On the one hand, checking satisfiability of $\textsc{Normalizes}_F$ can be used
to conclude the impossibility of a good matching scheme — at least for the
given language.

\begin{lemma}

  If $\textsc{Normalizes}_F$ is unsatisfiable, then there exists no simple
  matching scheme that normalizes $\mathcal{I}$ for the language $L$.

\end{lemma}

In this case, the reason for unsatisfiability can be used as a counterexample
to refine the given language, for example, by adding additional predicates.

\begin{example}

  TODO SHOW THAT THE REGISTER LANGUAGE EXAMPLE WITHOUT ONE OF THE PREDICATES
  WILL LEAD TO UNSAT.

\end{example}

On the other hand, the satisfiability of $\textsc{Normalizes}_F$ does not
necessarily lead to a unique matching scheme, since $\textsc{Normalize}_F$ can
have multiple satisfying assignments. Furthermore, $\textsc{Normalizes}_F$ does
not necessarily normalize $\mathcal{I}$. For one reason, $F$ may not be a
complete set of examples of executions and invalid projections. Second, our
simple characterization of $\textsc{Normalize}_F$ does not rule out the other
reasons for a given example $\tup{e,e'} \in F$ to be an invalid projection,
e.g.,~that the removed operations do not constitute \emph{multiple} matches.
However, we believe that checking satisfiability of $\textsc{Normalize}_F$ is
useful nonetheless: at the very least, satisfying assignments can be used as
\emph{assistance} in constructing normalizing matching schemes.
