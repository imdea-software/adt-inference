\section{Computing Matching Schemes}
\label{sec:matching}

TODO MOTIVATE THIS SECTION

When $P$ is a $k$-length sequence of $n$-ary predicates, we write $P(x_1,
\ldots, x_n)$ to denote the sequence
\begin{align*}
  P_1(x_1, \ldots, x_n) \ldots P_k(x_1, \ldots, x_n)
\end{align*}
Given finite predicate sequences $P$ and $Q$ and a set $E$ of executions, a
matching scheme $M$ is \emph{simple} when there exists an $n$-ary Boolean
function $G$, for $n = 2\cdot|P|+|Q|$, such that for each execution $e \in E$
\begin{itemize}

  \item $G(P(e,o_1),P(e,o_2),Q(e,o_1,o_2))$ is satisfied for at most one
  operation $o_1$, for any operation $o_2$,

  \item $M(e)(o_2)$ is undefined unless there exists an operation $o_1$
  for which $G(P(e,o_1),P(e,o_2),Q(e,o_1,o_2))$, and

  \item $M(e)(o_2) = o_1$ if{f} $G(P(e,o_1),P(e,o_2),P(e,o_1,o_2))$,

\end{itemize}
where $o_1$ and $o_2$ range over the operations of $e$.

TODO QUALIFY THE EXECUTION SETS, E.G., TO HAVE UNIQUE VALUES, AND SO ON

\begin{example}

  TODO SHOW A FINITARY SIMPLE MATCHING SCHEME
  
\end{example}

\begin{definition}

  The \emph{matching scheme inference problem} is to compute a faithful, simple
  matching scheme $M$ for a given set $E$ of executions and finite predicate
  sequences $P$ and $Q$.

\end{definition}

\begin{theorem}

  The matching scheme inference problem is computable for \ldots

\end{theorem}
