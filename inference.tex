%!TEX root = draft.tex
\section{The Symbolic ADT Inference Problem}
\label{sec:inference}

In this section we formalize a notion of abstract data type and define the
corresponding refinement and inference problems. These are the foundational
problems addressed in this work.

A \emph{kernel} of a set $H$ is a minimal set $K$ generating $H$ via the
relation $\mathord\to = (\to_\mathrm{o} \cup \to_\mathrm{p} \cup
\to_\mathrm{c})$, i.e.,
\begin{align*}
  H = \set{ h' : \exists h \in K.\ h \to^\ast h' }.
\end{align*}
An \emph{abstract data type (ADT)} $A$ is the kernel of the set $H(\mathcal{I})$
of histories of some implementation $\mathcal{I}$, and $H(A)$ denotes the
closure of $A$ under $\to$. While the kernels of arbitrary-width history sets
may not be unique, those with \emph{atomic} kernels, i.e.,~of width-$1$, are
guaranteed to be unique. Section~\ref{sec:nature} demonstrates that the
histories of naturally-occurring implementations have bounded-width and
unique ADTs, which we assume for the remainder of this work.

We define \emph{observational refinement} between implementations and ADTs as
history-set inclusion. Although this refinement is typically defined with
respect to the admissibility of program executions, recent work demonstrates
that these definitions are equivalent~\cite{conf/popl/BouajjaniEEH15}.

\begin{definition}

  An implementation $\mathcal{I}_1$ \emph{refines} another implementation
  $\mathcal{I}_2$ when $H(\mathcal{I}_1) \subseteq H(\mathcal{I}_2)$. An
  implementation $\mathcal{I}$ refines an ADT $A$ when $\mathcal{I}$ refines
  $H(A)$.

\end{definition}

Recent works demonstrate efficient algorithms\footnote{In time polynomial in
the number of operations, per execution.} for checking observational
refinement~\cite{conf/popl/BouajjaniEEH15, conf/pldi/EmmiEH15}, yet rely on
hand-written symbolic ADT representations. In order to frame the problem of
computing these automatically, we fix a language for symbolic representation. A
\emph{history formula} is a first-order logic formula with
\begin{itemize}

  \item variables ranging over operation identifiers,

  \item constants from $\mathbb{M}$ for method names,

  \item function symbols ${\sf f}$ and ${\sf m}$ for labels and matching, and

  \item predicate symbols ${\sf c}$, ${\sf um}$, {\sf r}, and ${\sf <}$ for completion,
  non-matching (operations which are not in the domain of the matching function), read-only, and order.

\end{itemize}
A history formula $F$ is interpreted over a history $h$ in the natural way, by
binding variables to the operations of $h$, and binding function and predicate
symbols to their interpretations in $h$. We write $h \models F$ when $h$ is a
model of $F$, and $h \not\models F$ otherwise.

\begin{example}

  The following history formula is satisfied by histories in which no write
  operation happens between a pair of matching write and read operations:
  \begin{align*}
    \forall x_1, x_2, x_3.\
    & {\sf c}(x_1) \land {\sf f}(x_1) = \mathrm{write} \land {\sf m}(x_1) = x_1 \\
    & \land {\sf c}(x_2) \land {\sf f}(x_2) = \mathrm{write} \land {\sf m}(x_2) = x_2 \\
    & \land {\sf c}(x_3) \land {\sf f}(x_3) = \mathrm{read} \land {\sf m}(x_3) = x_1 \\
    & \land x_1 < x_2
      \implies x_3 < x_2
  \end{align*}
  This is one of several requirements of atomic single-value register ADTs.
  It is satisfied by certain linearizations of the histories $H(e_2)$ and
  $H(e_3)$ from Example~\ref{ex:histories-of-executions}, yet not $H(e_1)$.

\end{example}

The \emph{bounded complement} of a history set $H$ of width $k \in \mathbb{N} \cup
\set{\omega}$ is the set of histories of width at most $k$ which are excluded
from $H$. Let $A$ be an ADT and $B$ its bounded complement. We say that
a history formula $F$ \emph{represents} $A$ when
\begin{itemize}

  \item $h \models F$ for all $h \in A$, and

  \item $h \not\models F$ for all $h \in B$.

\end{itemize}
ADT inference is to compute a formula representing an ADT.

\begin{definition}

  The \emph{symbolic ADT inference problem} is to compute a history formula
  representing the ADT $\ker H(\mathcal{I})$ of a given (reference) implementation
  $\mathcal{I}$.

\end{definition}

Computing a history formula representing the ADT $\ker H(\mathcal{I})$ of a
reference implementation $\mathcal{I}$ thus enables the efficient modular
program reasoning without the burden of writing precise formal specifications.
