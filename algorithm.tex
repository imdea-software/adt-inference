%!TEX root = draft.tex
\section{A Symbolic ADT Inference Algorithm}
\label{sec:algorithm}

Algorithm~\ref{alg:inference} solves the symbolic ADT inference problem by
computing a finite representation of an ADT complement $B$ using the $\preceq$
relation of Section~\ref{sec:patterns}. In general, this is achieved by
enumerating the elements of $B$ while maintaining the $\preceq$-minimal
elements, and recognizing a condition under which all the elements of $B$ are
related to the current set of minimals~\cite{conf/lics/AbdullaCJT96,
journals/tcs/FinkelS01}. In our case, we stratify our enumeration of $B$ by the
relative sizes of its histories. Formally, we say the \emph{weight} of a history
is the maximum among operation frequencies and the number of matches. We then
define
\begin{align*}
  & B_i = \set{ h \in B : \weight(h) \le i } \\
  & B_i' = \set{ h \in B : \exists h' \in B_i.\ h' \preceq h \text{ and } \weight(h) \le i+1 }
\end{align*}
respectively as the histories of $B$ with at most $i$ matches and duplicates,
and those derived from $B_i$ with at most $i+1$ matches and duplicates. We say
that an ADT with complement $B$ is \emph{predictable} if $B_i^* = B$ whenever
$B_i' = B_{i+1}$, i.e.,~if all histories of $B$ are represented by $B_i$ whenever
all histories of $B_{i+1}$ are represented by $B_i$. Algorithm~\ref{alg:inference}
then performs a weight-increasing enumeration of $B$, collecting
$\preceq$-minimals from smaller-weight violations before advancing to greater
weights. When no violation is found at a given weight, the algorithm terminates.
This algorithm is guaranteed to terminate since $\preceq$ is a
wqo~\cite{conf/lics/AbdullaCJT96, journals/tcs/FinkelS01}. Furthermore, this
algorithm is sound for arbitrary ADTs, and complete for predictable ADTs.
Many naturally-occurring ADTs are predictable — in fact all of the examples we
know of are predictable, as demonstrated in Section~\ref{sec:nature}.

\begin{algorithm}[t]
  \SetAlgoLined
  \SetKwInOut{Input}{Input}
  \Input{A reference implementation $\mathcal{I}$ of width $k$}
  \KwResult{A formula representing the ADT of $\mathcal{I}$}
  patterns $\gets$ $\emptyset$ \;
  $w$ $\gets$ $0$ \;
  \Repeat{none-found}{
    none-found $\gets$ true \;
    \For{each $k$-width history $h$ with weight $w$}{
      \uIf{$h$ is \emph{executable} with $\mathcal{I}$}{
        continue
      }
      \uElseIf{$\exists h' \in \text{patterns}.\ h' \preceq h$}{
        continue
      }
      \Else{
        add $h$ to patterns \;
        none-found $\gets$ false
      }
    }
    $w$ $\gets$ $w$ + $1$ \;
  }
  \Return $F(\text{patterns})$
  \caption{Symbolic ADT inference.}
  \label{alg:inference}
\end{algorithm}

\begin{theorem}
  \label{thm:algorithm}

  Algorithm~\ref{alg:inference} terminates. If the input implementation’s
  ADT is predictable, then the returned formula represents it.

\end{theorem}

\begin{proof}
  Termination is a direct consequence of Lemma~\ref{lem:wqo}, and since the
  number of histories of any weight $i \in \mathbb{N}$ is finite\footnote{
    As noted in Section~\ref{sec:histories}, we consider equality between
    histories up to renaming of operation identifiers.
  }
  each $B_i$ is computable.
  When $A$ is predictable, then $B = \text{patterns}^*$, and thus by
  Lemma~\ref{lem:formula1}, the returned formula $F(\text{patterns})$
  represents $A$.
\end{proof}

For non-predictable ADTs, the value of $\text{patterns}^*$ upon termination of
Algorithm~\ref{alg:inference} is an under-approximation of the bounded ADT
complement $B$. The resulting symbolic ADT representation is still satisfied by
all histories in $A$, therefore it would still be sound for modular program
reasoning, identifying only actual violations, and implying the correctness of
client programs which do not depend on the stronger criteria which excludes
unidentified violations. However, the under-approximation may be incomplete in
identifying all violations, and in proving client programs which depend on the
stronger criteria which excludes them all.
