%!TEX root = draft.tex
\section{Computing Symbolic Representations}
\label{sec:algorithm}

TODO MOTIVATE THIS SECTION

In addition to the relations $\to_\mathrm{o}$, $\to_\mathrm{p}$, and
$\to_\mathrm{c}$ of Section~\ref{sec:histories} under which all ADTs are
closed, the ADTs we consider in this work are also closed under additional
relations which remove matches, and duplicate operations. The relations
$\to_\mathrm{m}$ and $\to_\mathrm{d}$ relate two histories $h_1$ and $h_2$ when
$h_2$ is obtained from $h_1$ by
\begin{itemize}

  \item removing a match (m), or

  \item removing a duplicate operation (d).

\end{itemize}
We say an ADT closed under $\to_\mathrm{m}$ and $\to_\mathrm{d}$ is
\emph{normal}. In Section~\ref{sec:nature} we demonstrate that
naturally-occurring ADTs are normal. Defining the relation $\succeq$ as the
reflexive and transitive closure of the above relations,
\begin{align*}
  \succeq = (\to_\mathrm{o} \cup \to_\mathrm{p} \cup \to_\mathrm{c} \cup 
  \to_\mathrm{m} \cup \to_\mathrm{d})^\ast
\end{align*}
closure under $\succeq$ follows immediately.

\begin{lemma}

  Normal ADTs are closed under $\succeq$.

\end{lemma}

TODO MOTIVATE WQO

A \emph{well-quasi-ordering (wqo)} $R$ on a set $X$ is a reflexive, transitive
binary relation on $X$ for which in every infinite sequence $x_0 x_1 \ldots$ of
elements from $X$, there exists indices $i < j$ such that $R(x_i,x_j)$.

\begin{example}

  Consider the infinite history sequence $h_1 h_2 \ldots$ where each $h_i$
  contains $2i$ operations $o_1, o_1', \ldots, o_i, o_i'$ where each $o_j$ is a
  completed {\tt push} operation matching itself, and each $o_j'$ is a pending
  {\tt pop} operation with undefined matching. Because each successive $h_i$
  has both more matches and more pending operations, no two histories of the
  sequence are related by $\preceq$. Thus $\preceq$ is not a wqo.

\end{example}

The previous example demonstrates that $\preceq$ is not a wqo by allowing each
history $h_i$ of the infinite sequence to contain more and more pending
operations in order to ensure that $h_j \not\preceq h_i$ for every $j < i$.
Curbing this ability by limiting the maximum amount of pending operations per
history makes $\preceq$ a wqo. Although limiting to $k$ pending operations
essentially limits us to width-$k$ histories, of executions with at most $k$
operations parallel at any moment, e.g.,~of programs with at most $k$ threads,
this restriction comes at no loss of completeness when considering only the
histories of bounded-width ADT kernels, which is the subject of the remainder
of this section.

\begin{lemma}

  $\preceq$ is a wqo on width-bounded histories.

\end{lemma}

For the remainder of this section, we fix an ADT $A$, and let $B$ be the
complement of $\ker A$. When $\ker A$ has bounded width, so does $B$, and thus
$\preceq$ is a wqo on $B$. Furthermore, when $A$ normal, it is closed under
$\succeq$, and thus $B$ is closed under $\preceq$. Closure under a wqo implies
representation by a finite set. Formally, we say a set $X$ is \emph{finitely
representable} if there exists a finite set $Y$ and a relation $R \subseteq Y
\times X$ such that $X = \set{ x : \exists y\in Y.\ R(y,x) }$. In our case, we
obtain a finite set from which exactly the elements of $B$ are related by
$\preceq$.

\begin{lemma}

  $B$ is finitely representable if $A$ is normal.

\end{lemma}

Computing $B$ thus amounts to computing its finite representation. In general,
this is achieved by enumerating the elements of $B$ while maintaining the
$\preceq$-minimal elements, and recognizing a condition under which all
elements of $B$ are related to the current set of
minimals~\cite{conf/lics/AbdullaCJT96, journals/tcs/FinkelS01}. In our case, we
recognize a condition on the weight-increasing enumeration of $B$ which
guarantees coverage by the current set of minimals. Formally, for all $i \in
\mathbb{N}$, we define
\begin{align*}
  B_i &= \set{ h \in B : \weight(h) \le i } \\
  B_i' &= \set{ h' \in B :
    \exists h \in B_i.\ h \preceq h' \text{ and } \weight(h') \le i+1
  }
\end{align*}
as, respectively, the histories of $B$ with at most $i$ matches and duplicates,
and those derived from $B_i$ with at most $i\!+\!1$ matches and duplicates. We
say that $A$ is \emph{predictable} if $B_i = B$ whenever $B_i' = B_{i+1}$. In
Section~\ref{sec:nature} we demonstrate that naturally-occurring ADTs are
predictable. Since the number of weight-$i$ histories is finite\footnote{As
noted in Section~\ref{sec:histories}, we consider equality between histories up
to renaming of operation identifiers.} for each $i \in \mathbb{N}$, each $B_i$
is computable.

\begin{lemma}

  $B_i$ is computable, for all $i \in \mathbb{N}$, if $A$ is normal.

\end{lemma}

Predictability ensures that once $B_{i+1}$ does not alter the set of computed
$\preceq$-minimals, with respect to those of $B_i$, then the minimals computed
for $B_i$ are exactly those of $B$.

\begin{lemma}

  $B$ is computable if $A$ is normal and predictable.

\end{lemma}

Note that for non-predictable ADTs, computing $B_i$, for any $i \in
\mathbb{N}$, under-approximates $B$. In practice, computing $B_i$ can be used
to identify all violations which surface in histories with $i$ matches,
overlooking violations which only surface with more than $i$ matches. The
resulting symbolic ADT representation would still be sound in only identifying
\emph{actual} violations, yet incomplete in identifying \emph{all} violations.

WRITE A CLEAN SEGUE TO THE NEXT PART

Let $h = \tup{O,<,c,f,m}$ be a history with operations $O = \set{ o_1, \ldots,
o_n }$. Without loss of generality, we assume the match targets $\set{o_1,
\ldots, o_k}$ are indexed consecutively from $1$ to $k$, for some $k < n$. We
define the macro $\textsc{Embed}_o(x,Y,z)$:
\begin{align*}
  & (c(o) \implies {\sf c}(x)) \land {\sf f}(x) = f(o) \\
  & \land (o \in \dom(m) \implies \lnot{\sf um}(x) \land {\sf m}(x) = z) \\
  & \land (o \not\in \dom(m) \implies {\sf um}(x)) \land \bigwedge_{y \in Y} x < y
\end{align*}
capturing the correspondence between the logical variable $x$ and operation
$o$. The variables $Y$ and $z$ represent the operations ordered after $x$, and
the operation which $x$ matches. The macro $\textsc{Identical}(x,y)$, defined as
\begin{align*}
  {\sf f}(x) = {\sf f}(y)
  \land {\sf c}(x) \Leftrightarrow {\sf c}(y)
  \land {\sf m}(x) = {\sf m}(y)
\end{align*}
captures whether the operations bound to $x$ and $y$ are identical.
To express the constraints among the matches of an embedded history, we define
the macro $\textsc{Match}(Y,z)$:
\begin{align*}
  \forall x.\ {\sf m}(x) = z \implies
  \textsc{ReadOnly}(x) \lor \bigvee_{y \in Y} \textsc{Identical}(x,y)
\end{align*}
which requires any operation which matches $z$ to be either read-only, or
identical to some operation in $Y$, which represents the operations of $h$
which match $z$. Finally, we express the embedding of $h$ with the macro
$\textsc{Embed}_h$:
\begin{align*}
  \exists x_1, \ldots, x_n.\ \bigwedge_{i=1}^{n} \textsc{Embed}_{o_i}(x_i,Y_i,z_i)
  \land \bigwedge_{i=1}^{k} \textsc{Match}(M_i, x_i)
\end{align*}
where $Y_i = \set{ x_j : o_i < o_j }$ are the variables corresponding to
operations ordered after $o_i$, and $z_i$ is the variable corresponding to
$m(o_i)$, if defined, and $M_i = \set{ x_j : m(o_j) = o_i }$ are the variables
corresponding to operations matching $o_i$.

\begin{lemma}

  $h_1 \models \textsc{Embed}_{h_2}$ if{f} $h_1 \succeq h_2$.

\end{lemma}

We obtain a formula representing an ADT $A$ by taking the conjunction of the
negative embedding formulæ $\set{ \lnot\textsc{Embed}_h : h \in H }$ from any
set $H$ that generates the complement of $A$.

\begin{lemma}

  The formula $\bigwedge_{h \in H} \neg \textsc{Embed}_h$ represents $A$,
  for any set $H$ which generates the compliment of $A$.

\end{lemma}

For normal and predictable ADTs, such sets are computable, and so our inference
problem is also computable.

\begin{theorem}

  The symbolic ADT inference problem is computable for normal and predictable ADTs.

\end{theorem}
