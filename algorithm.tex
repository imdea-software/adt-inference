%!TEX root = draft.tex
\section{Solving the Symbolic ADT Inference Problem}

\begin{figure}
\begin{lstlisting}
Min := $\emptyset$
while (true) {
  $B_i$ := $ \set{ h \in B : \weight(h) = i }$;
  MinTemp := Min $\cup$ Min($B_i$)
  if ( MinTemp = Min )
    break;
  Min := MinTemp
}
return $F$(Min);
\end{lstlisting}
\caption{The algorithm solving the symbolic ADT inference problem for a given ADT $A$, where $B$ denotes the bounded complement of $\ker A$. The procedure $\texttt{Min}(X)$ returns the $\preceq$-minimal elements of the set of histories $X$. %and $\texttt{Formula}(X)$ returns a history formula which holds exactly for the histories in the upward closure of $X$.
} 
\label{fig:algorithm}
\end{figure}

Figure~\ref{fig:algorithm} lists the algorithm for solving the symbolic ADT inference problem which is based on the relation $\preceq$ 
defined in Section~\ref{sec:patterns}.
%For a given set of elements $X$ and an order relation $\preceq$, the \emph{upward closure}
%of $X$ w.r.t. $\preceq$ is the set $\uparrow X=\set{y : \exists x\in X.\ x\preceq y}$. 
The algorithm performs an weight-increasing enumeration of $B$ while maintaining the
$\preceq$-minimal elements, and stops whenever each history of weight $i$ is bigger (w.r.t. $\preceq$) than some
element of the collected set of minimals.

The partial correctness of the algorithm (i.e., ignoring termination) relies on the input ADT being \emph{predictable}, i.e.,
for all $i$,
\begin{align*}
B_{i+1} \succeq \texttt{Min}(B_{\leq i}) \mbox{ implies } \texttt{Min}(B) = \texttt{Min}(B_{\leq i}),
\end{align*}
where $B_{\leq i}=\bigcup_{1\leq j\leq i}B_j$.
Section~\ref{sec:nature} shows that many naturally-occurring ADTs are predictable.
%Although we don't provide a general result concerning predictability, we prove that it holds for  (see ).

\begin{theorem}\label{th:corr1}
For any predictable ADT $A$, if the algorithm in Figure~\ref{fig:algorithm} terminates, then the returned formula represents $A$.
\end{theorem}
\begin{proof}
%For a set $\Sigma$, let $\neg \Sigma$ denote its complement.
Let $M$ be the value of \texttt{Min} at the end of the \texttt{while} loop.
%By Lemma~\ref{lem:ADT_closure}, the complement of $A$ is closed under $\preceq$. Since 
%$M\subseteq B\subseteq  \neg \ker A \subseteq \neg A$, the closure of $M$, denoted by $Cl(M)$, satisfies $Cl(M)\subseteq \neg A$. 
Since $A$ is predictable, we have that $B \subseteq Cl(M)$. % where $Cl(M)$ denotes the closure of 
Therefore, by Lemma~\ref{lem:formula1}, the returned formula represents $A$.
\end{proof}

\begin{remark}
Note that for non-predictable ADTs, the value of $\texttt{Min}$ at the end of the \texttt{while} loop 
is an under-approximation of $\texttt{Min}(B)$. 
The resulting symbolic ADT representation is still satisfied by all histories in $\ker A$, therefore it would still be sound in only identifying
\emph{actual} violations, yet incomplete in identifying \emph{all} violations.
\end{remark}

The termination of the algorithm in Figure~\ref{fig:algorithm} is a direct consequence of Lemma~\ref{lem:wqo}.
Also, since the number of weight-$i$ histories is finite\footnote{As
noted in Section~\ref{sec:histories}, we consider equality between histories up
to renaming of operation identifiers.} for each $i \in \mathbb{N}$, each $B_i$
is computable.


\begin{theorem}\label{th:corr2}
The algorithm in Figure~\ref{fig:algorithm} terminates.
\end{theorem}





