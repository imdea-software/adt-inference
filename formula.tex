\section{Computing Symbolic Representations}
\label{sec:formula}

While Section~\ref{sec:patterns} demonstrates that (the complements) of
naturally-occurring ADTs can be represented finitely, in this section we
demonstrate that such finite representations have logical interpretations,
allowing us to derive the formulas representing ADTs. In what follows, we
describe how to obtain a formula which is satisfied by the histories which
embed a given history, via $\preceq$. Then, using the finite set of histories
which represent the complement of a given ADT, we represent the ADT itself as
the conjunction of negations of these embedding formulas. The resulting formula
is satisfied only on the histories which do not embed the generators of a given
ADT’s complement.

Let $h = \tup{O,<,c,f,m,r}$ be a history with operations $O = \set{ o_1, \ldots,
o_n }$. Without loss of generality, we assume the match targets $\set{o_1,
\ldots, o_k}$ of $h$ are indexed consecutively from $1$ to $k$, for some $k \le
n$. For each $o \in O$, we define the macro $\textsc{Embed}_o(x,Y,z)$:
\begin{align*}
  & \big(c(o) \implies {\sf c}(x)\big) \land {\sf f}(x) = f(o) \land {\sf r}(x) = r(o) \\
  & \land \big(o \in \dom(m)
    \implies \lnot{\sf um}(x) \land {\sf m}(x) = z\big) \\
  & \land \big(o \not\in \dom(m) \implies {\sf um}(x)\big)
    \land \bigwedge_{y \in Y} x < y
\end{align*}
capturing the correspondence between the operation $o$ and the logical variable
$x$ representing $o$. The variables $Y$ and $z$ represent the operations
ordered after $x$, and the operation which $x$ matches. The macro
$\textsc{Identical}(x,y)$:
\begin{align*}
  & \big({\sf c}(x) \Leftrightarrow {\sf c}(y)\big)
    \land {\sf f}(x) = {\sf f}(y) 
    \land \big({\sf r}(x) \Leftrightarrow {\sf r}(y)\big) \\
  & \land \big({\sf um}(x) \Leftrightarrow {\sf um}(y)\big)
    \land {\sf m}(x) = {\sf m}(y)
\end{align*}
captures whether the operations bound to $x$ and $y$ are identical.
To express the constraints among the matches of embedded operations, we define
the macro $\textsc{Match}(Y,z)$:
\begin{align*}
  \forall x.
  {\sf m}(x) = z \implies
  {\sf r}(x) \lor \bigvee_{y \in Y} \textsc{Identical}(x,y)
\end{align*}
which requires any operation which matches $z$ to be either read-only, or
identical to some operation in $Y$, which represents the operations of $h$
which match $z$. Finally, we express the embedding of $h$ with the macro
$\textsc{Embed}_h$:
\begin{align*}
  \exists x_1, \ldots, x_n.
  \bigwedge_{i=1}^{n} \textsc{Embed}_{o_i}(x_i,Y_i,z_i)
  \land \bigwedge_{i=1}^{k} \textsc{Match}(W_i, x_i)
\end{align*}
where $Y_i = \set{ x_j : o_i < o_j }$ are the variables corresponding to
operations ordered after $o_i$, and $z_i$ is the variable corresponding to
$m(o_i)$, if defined, and $W_i = \set{ x_j : m(o_j) = o_i }$ are the variables
corresponding to operations matching $o_i$.

\begin{example}
  \label{ex:formulas}

  Consider again the histories of Example~\ref{ex:patterns} which generate the
  complement of the atomic register ADT. The $\textsc{Embed}$ formula for the
  first history,
\begin{verbatim}
  [1:X] read => 1 (RO)  #
\end{verbatim}
  after simplifications, like replacing $\mathrm{true} \implies p$ with $p$, is
  \begin{align*}
    \exists x_1.\ {\sf c}(x_1) \land {\sf f}(x_1) = \mathrm{read} \land {\sf um}(x_1).
  \end{align*}
  The $\textsc{Embed}$ formula for second history,
\begin{verbatim}
  [1:1] write(1)            #
  [2:2] read => empty (RO)    #    
\end{verbatim}
  is similarly given by
  \begin{align*}
    & \exists x_1, x_2.\ x_1 < x_2 \\
    & \quad \land {\sf c}(x_1) \land {\sf f}(x_1) = \mathrm{write}
      \land \lnot{\sf um}(x_1) \land {\sf m}(x_1) = x_1 \\
    & \quad \land {\sf c}(x_2) \land {\sf f}(x_2) = \mathrm{read}
      \land \lnot{\sf um}(x_2) \land {\sf m}(x_2) = x_2 \\
    & \quad \land \big(\forall x.\ {\sf m}(x) = x_1 \implies {\sf r}(x) \lor \textsc{Identical}(x,x_1)\big) \\
    & \quad \land \big(\forall x.\ {\sf m}(x) = x_2 \implies {\sf r}(x) \lor \textsc{Identical}(x,x_2)\big).
  \end{align*}
  The formulas for the other histories are similarly obtained.

\end{example}

\begin{lemma}

  $h_1 \models \textsc{Embed}_{h_2}$ if{f} $h_1 \succeq h_2$.

\end{lemma}

We obtain a formula representing an ADT $A$ by taking the conjunction of the
negative embedding formulas $\set{ \lnot\textsc{Embed}_h : h \in H }$ from any
set $H$ that generates the complement of $A$.

\begin{lemma}

  The formula $\bigwedge_{h \in H} \neg \textsc{Embed}_h$ represents $A$,
  for any set $H$ which generates the compliment of $A$.

\end{lemma}

\begin{example}

  The conjunction of negations of the $\textsc{Embed}$ formula for the
  histories of Example~\ref{ex:patterns}, which are partially written in
  Example~\ref{ex:formulas}, represents the atomic register ADT.
  
\end{example}

For normal and predictable ADTs, such sets are computable, and so our inference
problem is also computable.

\begin{theorem}

  The symbolic ADT inference problem is computable for normal and predictable ADTs.

\end{theorem}
